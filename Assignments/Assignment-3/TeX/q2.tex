\section*{Question 2}

\paragraph{Given:}
LASSO regression primal problem:
\begin{equation}\label{eq:lasso-primal}
    \minimize_{\bfbeta \in \R^n}
    \quad
    \half \norm{\bfX \bfbeta - \y}_2^2
    + \lambda \norm{\bfbeta}_1,
    \qquad
    \bfX \in \R^{m \times n},
    \; \y \in \R^m,
    \; \lambda > 0
\end{equation}
with number of samples \( m \), number of features \( n \), and regularisation parameter \( \lambda > 0 \).

LASSO regression dual problem:
\begin{equation}\label{eq:lasso-dual}
    \maximize_{\u \in \R^m}
    \quad
    - \half \norm{\u}_2^2
    + \dotp{\y}{\u},
    \quad\qquad \subjectto \quad
    \norm{\dotp{\bfX}{\u}}_\infty \leq \lambda
\end{equation}

\subsection*{(1) Derivation of the dual problem~\peqref{eq:lasso-dual}}

The primal problem in \peqref{eq:lasso-primal} can be equivalently rewritten as
\begin{equation*}
    \minimize_{\bfbeta \in \R^n, \; \z \in \R^m}
    \quad
    \half \norm{\z}_2^2
    + \lambda \norm{\bfbeta}_1,
    \quad\qquad \subjectto \quad
    \z = \bfX \bfbeta - \y
\end{equation*}
by introducing an auxiliary variable \( \z \in \R^m \).

The Lagrangian function of the above constrained optimisation problem is given by
\begin{equation*}
    \func{\calL}{\bfbeta, \z, \bfmu}
    =
    \half \norm{\z}_2^2
    + \lambda \norm{\bfbeta}_1
    + \dotp{\bfmu}{\pbrac{\bfX \bfbeta - \y - \z}}
\end{equation*}
with Lagrange multiplier \( \bfmu \in \R^m \) for the constraint \( \z = \bfX \bfbeta - \y \).

The Lagrangian dual function of the above constrained optimisation problem is given by
\begin{equation*}
    \func{g}{\bfmu}
    =
    \inf_{\bfbeta \in \R^n, \; \z \in \R^m}
    \func{\calL}{\bfbeta, \z, \bfmu}
\end{equation*}
and the corresponding Lagrangian dual problem is given by
\begin{equation*}
    \maximize_{\bfmu \in \R^m}
    \quad
    \func{g}{\bfmu}
\end{equation*}

We can separate the infimum over \( \bfbeta \) and \( \z \) in the Lagrangian dual function as
\begin{equation*}
    \implies
    \func{g}{\bfmu}
    =
    \inf_{\bfbeta \in \R^n}
    \bbrac{
        \lambda \norm{\bfbeta}_1
        + \dotp{\bfmu}{\bfX \bfbeta}
    }
    +
    \inf_{\z \in \R^m}
    \bbrac{
        \half \norm{\z}_2^2
        - \dotp{\bfmu}{\z}
    }
    - \dotp{\bfmu}{\y}
\end{equation*}

The first term can be simplified to
\begin{equation*}
    \inf_{\bfbeta \in \R^n}
    \bbrac{
        \lambda \norm{\bfbeta}_1
        + \dotp{\pbrac{\dotp{\bfX}{\bfmu}}}{\bfbeta}
    }
    =
    \begin{cases}
        0,
        & \text{if }
        \norm{\dotp{\bfX}{\bfmu}}_\infty \leq \lambda
        \\
        -\infty, & \text{otherwise}
    \end{cases}
\end{equation*}
by the definition of the dual norm, since
\begin{equation*}
    \norm{\dotp{\bfX}{\bfmu}}_\infty
    =
    \sup_{\bfbeta \neq \zero}
    \frac
    {\dotp{\pbrac{\dotp{\bfX}{\bfmu}}}{\bfbeta}}
    {\norm{\bfbeta}_1}
\end{equation*}

The second term can be simplified to
\begin{equation*}
    \inf_{\z \in \R^m}
    \bbrac{
        \half \norm{\z}_2^2
        - \dotp{\bfmu}{\z}
    }
    =
    - \half \norm{\bfmu}_2^2
    \qquad
    \because
    \nabla_{\z} \,
    \bbrac{
        \half \norm{\z}_2^2
        - \dotp{\bfmu}{\z}
    }
    =
    \z - \bfmu
    \implies
    {\z}^\ast
    =
    \bfmu
\end{equation*}
by setting the gradient with respect to \( \z \) to zero.

Combining the above results, we have that the Lagrangian dual function is given by
\begin{equation*}
    \func{g}{\bfmu}
    =
    \begin{cases}
        - \half \norm{\bfmu}_2^2
        - \dotp{\bfmu}{\y},
        & \text{if }
        \norm{\dotp{\bfX}{\bfmu}}_\infty \leq \lambda
        \\
        -\infty, & \text{otherwise}
    \end{cases}
\end{equation*}

Thus, the LASSO regression dual problem is given by
\begin{equation*}
    \maximize_{\bfmu \in \R^m}
    \quad
    - \half \norm{\bfmu}_2^2
    + \dotp{\y}{\bfmu},
    \quad\qquad \subjectto \quad
    \norm{\dotp{\bfX}{\bfmu}}_\infty \leq \lambda
\end{equation*}
which is exactly the same as the dual problem in \peqref{eq:lasso-dual} with \( \u = \bfmu \).

\subsection*{(2) Strong duality of the LASSO regression problem}

The LASSO regression primal problem in \peqref{eq:lasso-primal} is a convex optimisation problem since the objective function is a sum of two convex functions (the squared \( \ell_2 \)-norm and the \( \ell_1 \)-norm) and there are no constraints.
Thus, to show that strong duality holds between the LASSO regression primal and dual problems, it suffices to show that the primal problem satisfies Slater's condition.
Slater's condition states that if there exists a strictly feasible point for a convex optimisation problem with inequality constraints, then strong duality holds.
In the case of the LASSO regression primal problem, there are no inequality constraints, so any point in \( \R^n \) is strictly feasible.
Therefore, Slater's condition is trivially satisfied, and strong duality holds between the LASSO regression primal and dual problems.
Hence, the optimal values of the primal and dual problems are equal, i.e.,
\begin{equation*}
    p^\ast = d^\ast
\end{equation*}
where \( p^\ast \) and \( d^\ast \) denote the optimal values of the primal and dual problems, respectively.

\subsection*{(5) Dual constraint \( \norm{\dotp{\bfX}{\bfmu}}_\infty \leq \lambda \) enforces sparsity in the primal solution \( \bfbeta^\ast \)}

The dual constraint \( \norm{\dotp{\bfX}{\bfmu}}_\infty \leq \lambda \) can be rewritten as
\begin{equation*}
    \abs{\dotp{\bfx_j}{\bfmu}} \leq \lambda,
    \qquad
    \forall j \in \set{1, 2, \ldots, n}
\end{equation*}
where \( \bfx_j \in \R^m \) denotes the \( j \)th column of the data matrix \( \bfX \).
From the complementary slackness condition, we have that for each \( j \),
\begin{equation*}
    \bfbeta_j^\ast \,\pbrac{
        \abs{\dotp{\bfx_j}{\bfmu^\ast}} - \lambda
    }
    = 0,
    \qquad
    \forall j \in \set{1, 2, \ldots, n}
\end{equation*}
where \( \bfbeta_j^\ast \) denotes the \( j \)th component of the primal optimal solution \( \bfbeta^\ast \) and \( \bfmu^\ast \) denotes the dual optimal solution.
This implies that if \( \abs{\dotp{\bfx_j}{\bfmu^\ast}} < \lambda \), then \( \bfbeta_j^\ast = 0 \).
Thus, the dual constraint \( \norm{\dotp{\bfX}{\bfmu}}_\infty \leq \lambda \) enforces sparsity in the primal solution \( \bfbeta^\ast \) by forcing certain components to be zero when the corresponding dual constraint is not tight.
