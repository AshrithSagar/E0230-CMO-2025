\section*{Question 3}

\subsection*{(1) Projections in a navigation problem}

\paragraph{Given:}
\begin{equation*}
    \calC_1
    =
    \set{\x \in \R^2 \given \norm{\x}_2 \leq 5},
    \qquad
    \calC_2
    =
    \set{\x \in \R^2 \given -3 \leq x_1 \leq 3, \; 0 \leq x_2 \leq 4}
\end{equation*}

The projections of a point \( \y \in \R^2 \) onto the convex sets \( \calC_1 \) and \( \calC_2 \) are given by
\begin{align*}
    \implies
    \func{\Pi_{\calC_1}}{\y}
    & =
    \begin{cases}
        \y, & \text{if } \norm{\y}_2 \leq 5
        \\
        5 \frac{\y}{\norm{\y}_2}, & \text{otherwise}
    \end{cases}
    \\
    \implies
    \func{\Pi_{\calC_2}}{\y}
    & =
    \pbrac[\Big]{
        \min{\pbrac{3, \max{\pbrac{-3, y_1}}}}
        , \;
        \min{\pbrac{4, \max{\pbrac{0, y_2}}}}
    } \in \R^2
\end{align*}

\begin{figure}[h!]
    \centering
    \includegraphics[width=0.85\textwidth]
    {../Codes/figures/Q3-1-Projections-24233.jpeg}% chktex 8
    \caption{
        Projections onto Circle \( \calC_1 \) and Box \( \calC_2 \)
    }
\end{figure}

\newpage
\subsection*{(2) Separating hyperplane in a classification story}

\paragraph{Given:}
\begin{equation*}
    \calC_A
    =
    \set{\x \in \R^2 \given \norm{\x}_2 \leq 1},
    \qquad
    \calC_B
    =
    \set{\x \in \R^2 \given x_1 \geq 3}
\end{equation*}

One separating hyperplane between the convex sets \( \calC_A \) and \( \calC_B \) can be seen as
\begin{equation*}
    \func{h}{\x}
    =
    \dotp{\bfw}{\x} + b
    =
    0,
    \quad
    \text{where }
    \bfw
    =
    \pbrac{1, 0}^\top,
    \quad
    b
    =
    -2
\end{equation*}

\begin{figure}[h!]
    \centering
    \includegraphics[width=\textwidth]
    {../Codes/figures/Q3-2-Separating-Hyperplane-24233.jpeg}% chktex 8
    \caption{
        Separating hyperplane between Group A (\( \calC_A \)) and Group B (\( \calC_B \)).
    }
\end{figure}

\subsection*{(3) Farkas lemma in a supply-chain model}

\paragraph{Given:}
System:
\begin{equation*}
    x_1 + x_2 \leq -1
    , \quad
    -x_1 \leq 0
    , \quad
    -x_2 \leq 0
\end{equation*}

The problem can be stated as an optimisation problem (more specifically, as a feasibility problem) as
\begin{equation*}
    \minimize_{\x \in \R^2}
    \;\; 0
    \qquad \subjectto \quad
    \begin{cases}
        x_1 + x_2 \leq -1
        \\
        -x_1 \leq 0
        \\
        -x_2 \leq 0
    \end{cases}
\end{equation*}
where we want to find if there exists any \( \x \in \R^2 \) that satisfies all the constraints.

The problem can be posed in a more general form as
\begin{equation*}
    \minimize_{\x \in \R^d}
    \;\; 0
    \qquad \subjectto \quad
    \A \x \leq \b,
    \qquad
    \A \in \R^{m \times d},
    \quad
    \b \in \R^m
\end{equation*}

The Lagrangian function for the above constrained optimisation problem is given by
\begin{equation*}
    \func{\calL}{\x, \bflambda}
    =
    0
    + \dotp{\bflambda}{\pbrac{\A \x - \b}}
    =
    \dotp{\bflambda}{\A \x}
    - \dotp{\bflambda}{\b}
\end{equation*}
and the corresponding Lagrangian dual function is given by
\begin{equation*}
    \func{g}{\bflambda}
    =
    \inf_{\x \in \R^d}
    \func{\calL}{\x, \bflambda}
    =
    \inf_{\x \in \R^d}
    \pbrac[\Big]{
        \dotp{\bflambda}{\A \x}
        - \dotp{\bflambda}{\b}
    }
    =
    \begin{cases}
        - \dotp{\bflambda}{\b}, & \text{if } \A^\top \bflambda = \zero
        \\
        -\infty, & \text{otherwise}
    \end{cases}
\end{equation*}
since \( \dotp{\pbrac{\dotp{\A}{\bflambda}}}{\x} - \dotp{\bflambda}{\b} \) is linear in \( \x \) and is unbounded below unless \( \A^\top \bflambda = \zero \).

Thereby, the Lagrangian dual problem can be stated as
\begin{equation*}
    \maximize_{\bflambda \in \R^m}
    \quad
    - \dotp{\bflambda}{\b}
    \qquad \subjectto \quad
    \dotp{\A}{\bflambda} = \zero,
    \quad
    \bflambda \geq \zero
\end{equation*}
which is equivalent to the problem stated in Farkas lemma, aiming to find a dual certificate of infeasibility, i.e., a \( \bflambda \geq \zero \) such that \( \dotp{\A}{\bflambda} = \zero \) and \( \dotp{\bflambda}{\b} < 0 \).

\paragraph{Meaning of \( \y \) in the supply-chain context}

In the supply-chain context, the Farkas certificate \( \y \geq \zero \) represents a combination of the system's capacity constraints that proves the infeasibility of the demand requirements.
Specifically, \( \y \) can be interpreted as a set of weights assigned to each constraint, indicating how much each constraint contributes to the overall infeasibility of the system.
If such a \( \y \) exists, it implies that the current supply-chain configuration cannot meet the demands without violating at least one of the capacity constraints, thus highlighting the need for adjustments in supply, logistics, or resource allocation to achieve feasibility.
