\section*{Question 3}

\subsection*{(1) Projections in a navigation problem}

\paragraph{Given:}
\begin{equation*}
    \calC_1
    =
    \set{\x \in \R^2 \given \norm{\x}_2 \leq 5},
    \qquad
    \calC_2
    =
    \set{\x \in \R^2 \given -3 \leq x_1 \leq 3, \; 0 \leq x_2 \leq 4}
\end{equation*}

The projections of a point \( \y \in \R^2 \) onto the convex sets \( \calC_1 \) and \( \calC_2 \) are given by
\begin{align*}
    \implies
    \func{\Pi_{\calC_1}}{\y}
    & =
    \begin{cases}
        \y, & \text{if } \norm{\y}_2 \leq 5
        \\
        5 \frac{\y}{\norm{\y}_2}, & \text{otherwise}
    \end{cases}
    \\
    \implies
    \func{\Pi_{\calC_2}}{\y}
    & =
    \pbrac[\Big]{
        \min{\pbrac{3, \max{\pbrac{-3, y_1}}}}
        , \;
        \min{\pbrac{4, \max{\pbrac{0, y_2}}}}
    } \in \R^2
\end{align*}

\begin{figure}[h!]
    \centering
    \includegraphics[width=0.85\textwidth]
    {../Codes/figures/Q3-1-Projections-24233.jpeg}% chktex 8
    \caption{
        Projections onto Circle \( \calC_1 \) and Box \( \calC_2 \)
    }
\end{figure}

\newpage
\subsection*{(2) Separating hyperplane in a classification story}

\paragraph{Given:}
\begin{equation*}
    \calC_A
    =
    \set{\x \in \R^2 \given \norm{\x}_2 \leq 1},
    \qquad
    \calC_B
    =
    \set{\x \in \R^2 \given \bbrac{\x}_1 \geq 3}
\end{equation*}
where the notation \( \bbrac{\x}_i \) denotes the \( i \)-th component of the vector \( \x \).

We can see that both \( \calA \) and \( \calB \) are convex sets in \( \R^2 \) since
\vspace{-0.5em}
\begin{align*}
    &
    \forall \x_1, \x_2 \in \calC_A,
    \quad
    \forall \theta \in [0, 1],
    \quad
    \\ &
    \implies
    \norm{\theta \x_1 + (1 - \theta) \x_2}_2
    \leq
    \theta \norm{\x_1}_2
    + (1 - \theta) \norm{\x_2}_2
    \leq
    \theta \cdot 1
    + (1 - \theta) \cdot 1
    = 1
    \\ &
    \therefore
    \quad
    \theta \x_1 + (1 - \theta) \x_2 \in \calC_A
    \\[5pt] &
    \forall \x_1, \x_2 \in \calC_B,
    \quad
    \forall \theta \in [0, 1],
    \quad
    \\ &
    \implies
    \bbrac[\Big]{\theta \x_1 + (1 - \theta) \x_2}_1
    =
    \theta \bbrac{\x_1}_1 + (1 - \theta) \bbrac{\x_2}_1
    \geq
    \theta \cdot 3 + (1 - \theta) \cdot 3
    = 3
    \\ &
    \therefore
    \quad
    \theta \x_1 + (1 - \theta) \x_2 \in \calC_B
\end{align*}

\vspace{-0.5em}
We can see that \( \calC_A \) and \( \calC_B \) are disjoint due to the relation \( \bbrac{\x}_1 \leq \norm{\x}_2, \; \forall \x \in \R^2 \), giving
\vspace{-0.5em}
\begin{align*}
    \forall \x \in \calC_A,
    \quad
    \bbrac{\x}_1 \leq \norm{\x}_2 \leq 1
    \quad \implies \quad
    \x \notin \calC_B
    \\[5pt]
    \forall \x \in \calC_B,
    \quad
    3 \leq \bbrac{\x}_1 \leq \norm{\x}_2
    \quad \implies \quad
    \x \notin \calC_A
\end{align*}

\vspace{-0.5em}
Since both \( \calC_A \) and \( \calC_B \) are convex and disjoint sets, by the separating hyperplane theorem, which states that for any two non-empty, disjoint convex sets, there exists at least one hyperplane that separates them, we can conclude that there exists a hyperplane that separates the two sets \( \calC_A \) and \( \calC_B \).

One way to find such a hyperplane is to consider the optimisation problem of minimising the distance between points in the two sets, i.e.,
\vspace{-0.5em}
\begin{equation*}
    \minimize_{\x_A \in \calC_A, \; \x_B \in \calC_B}
    \quad
    \norm{\x_A - \x_B}_2
\end{equation*}
which can be solved using CVXPY, representing \( \calC_A \) and \( \calC_B \) using the respective constraints.

The optimal solution to this problem will give us the closest points \( {\x}^\ast_A \in \calC_A \) and \( {\x}^\ast_B \in \calC_B \).
The hyperplane that separates the two sets can then be seen as being perpendicular to the line segment connecting \( {\x}^\ast_A\) and \( {\x}^\ast_B \) and passes through any interior point on this line segment, which for simplicity, we can take the midpoint \( \frac{1}{2} \pbrac{ {\x}^\ast_A + {\x}^\ast_B } \).

For the above problem, the optimal solutions can be seen to be
\vspace{-0.5em}
\begin{equation*}
    {\x}^\ast_A
    =
    \pbrac{1, 0}^\top,
    \quad
    {\x}^\ast_B
    =
    \pbrac{3, 0}^\top
\end{equation*}

\vspace{-0.5em}
Thereby, the normal vector of the separating hyperplane with respect to \( \calC_A \) is given by \( \bfn = {\x}^\ast_B - {\x}^\ast_A = \pbrac{2, 0}^\top \), and the midpoint on the line segment connecting \( {\x}^\ast_A \) and \( {\x}^\ast_B \) is given by \( \bfm = \frac{1}{2} \pbrac{ {\x}^\ast_A + {\x}^\ast_B } = \pbrac{2, 0}^\top \).
Thereby, one such separating hyperplane between the convex sets \( \calC_A \) and \( \calC_B \) can be seen as
\vspace{-0.5em}
\begin{equation*}
    \func{h}{\x}
    =
    \dotp{\bfw}{\x} + b
    =
    0,
    \quad
    \text{where }
    \bfw
    =
    \hat{\bfn}
    =
    \pbrac{1, 0}^\top,
    \quad
    b
    =
    - \dotp{\bfw}{\bfm}
    =
    -2
\end{equation*}

\( \therefore \) The separating hyperplane between \( \calC_A \) and \( \calC_B \) is given by \( \boxed{ \set{\x \in \R^2 \given \bbrac{\x}_1 = 2} } \).

\begin{figure}[h!]
    \centering
    \includegraphics[width=\textwidth]
    {../Codes/figures/Q3-2-Separating-Hyperplane-24233.jpeg}% chktex 8
    \caption{
        Separating hyperplane between Group A (\( \calC_A \)) and Group B (\( \calC_B \)).
    }
\end{figure}

\clearpage
\subsection*{(3) Farkas lemma in a supply-chain model}

\paragraph{Given:}
System:
\begin{equation*}
    x_1 + x_2 \leq -1
    , \quad
    -x_1 \leq 0
    , \quad
    -x_2 \leq 0
\end{equation*}

The problem can be stated as an optimisation problem (more specifically, as a feasibility problem) as
\begin{equation*}
    \minimize_{\x \in \R^2}
    \;\; 0
    \qquad \subjectto \quad
    \begin{cases}
        x_1 + x_2 \leq -1
        \\
        -x_1 \leq 0
        \\
        -x_2 \leq 0
    \end{cases}
\end{equation*}
where we want to find if there exists any \( \x \in \R^2 \) that satisfies all the constraints.

The problem can be posed in a more general form as
\begin{equation*}
    \minimize_{\x \in \R^d}
    \;\; 0
    \qquad \subjectto \quad
    \A \x \leq \b,
    \qquad
    \A \in \R^{m \times d},
    \quad
    \b \in \R^m
\end{equation*}

The Lagrangian function for the above constrained optimisation problem is given by
\begin{equation*}
    \func{\calL}{\x, \bflambda}
    =
    0
    + \dotp{\bflambda}{\pbrac{\A \x - \b}}
    =
    \dotp{\bflambda}{\A \x}
    - \dotp{\bflambda}{\b}
\end{equation*}
and the corresponding Lagrangian dual function is given by
\begin{equation*}
    \func{g}{\bflambda}
    =
    \inf_{\x \in \R^d}
    \func{\calL}{\x, \bflambda}
    =
    \inf_{\x \in \R^d}
    \pbrac[\Big]{
        \dotp{\bflambda}{\A \x}
        - \dotp{\bflambda}{\b}
    }
    =
    \begin{cases}
        - \dotp{\bflambda}{\b}, & \text{if } \A^\top \bflambda = \zero
        \\
        -\infty, & \text{otherwise}
    \end{cases}
\end{equation*}
since \( \dotp{\pbrac{\dotp{\A}{\bflambda}}}{\x} - \dotp{\bflambda}{\b} \) is linear in \( \x \) and is unbounded below unless \( \A^\top \bflambda = \zero \).

Thereby, the Lagrangian dual problem can be stated as
\begin{equation*}
    \maximize_{\bflambda \in \R^m}
    \quad
    - \dotp{\bflambda}{\b}
    \qquad \subjectto \quad
    \dotp{\A}{\bflambda} = \zero,
    \quad
    \bflambda \geq \zero
\end{equation*}
which is equivalent to the problem stated in Farkas lemma~\citep{Nocedal2006}, aiming to find a dual certificate of infeasibility, i.e., a \( \bflambda \geq \zero \) such that \( \dotp{\A}{\bflambda} = \zero \) and \( \dotp{\bflambda}{\b} < 0 \).

\newpage
We solve the dual problem using CVXPY to obtain the Farkas certificate \( \bflambda^\ast \) as

\begin{verbatim}
Is feasible: False
Certificate y: [9.5163699e-10 9.5163699e-10 9.5163699e-10]
Diagnostics: {
'primal_status': 'infeasible',
'primal_objective_value': inf,
'primal_solver_stats': SolverStats(
    solver_name='CLARABEL',
    solve_time=2.9209e-05,
    setup_time=None,
    num_iters=5,
    extra_stats=None
),
'dual_status': 'optimal',
'dual_objective_value': np.float64(-9.516369896651987e-10),
'dual_solver_stats': SolverStats(
    solver_name='CLARABEL',
    solve_time=2.2418e-05,
    setup_time=None,
    num_iters=5,
    extra_stats=None
),
'dual_constraints_residual': array([-9.47221063e-20, -9.47221063e-20]),
'dual_constraints_residual_norm': np.float64(1.3395728739261362e-19)
}
\end{verbatim}

As can be seen from the values, the \verb|dual_constraints_residual| which is \( - \dotp{\bflambda}{\b} \) is indeed negative, and we've found a Farkas certificate \( \y = \bflambda^\ast \geq \zero \) such that \( \dotp{\A}{\y} = \zero \) and \( \dotp{\y}{\b} < 0 \), thereby proving that the original system of inequalities is infeasible.

\paragraph{Meaning of \( \y \) in the supply-chain context}

In the supply-chain context, the Farkas certificate \( \y \geq \zero \) represents a combination of the system's capacity constraints that proves the infeasibility of the demand requirements.
Specifically, \( \y \) can be interpreted as a set of weights assigned to each constraint, indicating how much each constraint contributes to the overall infeasibility of the system.
If such a \( \y \) exists, it implies that the current supply-chain configuration cannot meet the demands without violating at least one of the capacity constraints, thus highlighting the need for adjustments in supply, logistics, or resource allocation to achieve feasibility.
