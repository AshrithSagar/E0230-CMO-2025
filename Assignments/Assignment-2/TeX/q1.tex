\section*{Question 1: Part 1 | Q-Gram-Schmidt \& Conjugate Direction method}

\paragraph{Given:}
\( f: \R^d \to \R \)
\begin{equation*}
    \func{f}{\x}
    =
    \half \qf{\x}{\Q}
    - \dotp{\b}{\x},
    \quad \Q \in \PD, \b \in \R^d
\end{equation*}

\( \Q \)-Gram-Schmidt recursion:
\begin{equation}
    \d_{k+1}
    =
    \p_{k+1}
    - \sum_{i = 0}^{k} \frac{\qf{\p_{k+1}}{\Q}[\d_i]}{\qf{\d_i}{\Q}} \d_i
\end{equation}

\subsection*{(a) \( \Q \)--conjugacy of \( \set{\d_k} \)}

Yes, the vectors produced by the above recursion are \( \Q \)--conjugate, which can be shown by induction.

\paragraph{Base case:}
\( k = 0 \)
\begin{align*}
    \implies
    \d_1
    & =
    \p_1
    - \frac{\qf{\p_1}{\Q}[\d_0]}{\qf{\d_0}{\Q}} \d_0
    \\
    \implies
    \qf{\d_1}{\Q}[\d_0]
    & =
    \qf{\p_1}{\Q}[\d_0]
    - \frac{\qf{\p_1}{\Q}[\d_0]}{\cancel{\qf{\d_0}{\Q}}} \cancel{\qf{\d_0}{\Q}}
    =
    \qf{\p_1}{\Q}[\d_0]
    - \qf{\p_1}{\Q}[\d_0]
    =
    0
\end{align*}

Thus, \( \d_1 \) is \( \Q \)--conjugate to \( \d_0 \), and the base case holds.

\paragraph{Inductive step:}
Assume that \( \d_i \) is \( \Q \)--conjugate to \( \d_j \) for all \( i, j \leq k \), \( i \neq j \).
We need to show that \( \d_{k+1} \) is \( \Q \)--conjugate to \( \d_j \) for all \( j \leq k \).

For \( j \leq k \), we  as follows
\begin{align*}
    \implies
    \qf{\d_{k+1}}{\Q}[\d_j]
    & =
    \qf{\p_{k+1}}{\Q}[\d_j]
    - \sum_{i = 0}^{k} \frac{\qf{\p_{k+1}}{\Q}[\d_i]}{\qf{\d_i}{\Q}} \qf{\d_i}{\Q}[\d_j]
    \\ & =
    \qf{\p_{k+1}}{\Q}[\d_j]
    - \frac{\qf{\p_{k+1}}{\Q}[\d_j]}{\cancel{\qf{\d_j}{\Q}}} \cancel{\qf{\d_j}{\Q}}
    - \sum_{\substack{i = 0 \\ i \neq j}}^{k} \frac{\qf{\p_{k+1}}{\Q}[\d_i]}{\qf{\d_i}{\Q}} \cancelto{0}{\qf{\d_i}{\Q}[\d_j]}
    \\ & =
    \qf{\p_{k+1}}{\Q}[\d_j]
    - \qf{\p_{k+1}}{\Q}[\d_j]
    - 0
    = 0
\end{align*}

Thus, \( \d_{k+1} \) is \( \Q \)--conjugate to \( \d_j \) for all \( j \leq k \), and the inductive step holds.

Hence, the vectors \( \set{\d_k} \) produced by the above recursion are \( \Q \)--conjugate.

\subsection*{(b) \( \Q = \I \) case}

\( \Q \)--conjugacy reduces to orthogonality for \( \Q = \I \), and the above recursion reduces to the classical Gram-Schmidt orthogonalisation process.
\begin{equation*}
    \d_{k+1}
    =
    \p_{k+1}
    - \sum_{i = 0}^{k} \frac{\dotp{\p_{k+1}}{\d_i}}{\dotp{\d_i}{\d_i}} \d_i
\end{equation*}

\subsection*{(c) Conjugate Descent method}

\( \pbrac[\big]{\alpha_k, -\dotp{\grad{f}{\x_k}}{\u_k}, \lambda_k} \) for the first 7 iterations of \texttt{CD\_SOLVE}:

\begin{lstlisting}[firstnumber=0]
(-0.007849, 0.333892, 42.539809)
(-0.005127, 0.208773, 40.720898)
(0.006767, -0.262146, 38.741209)
(0.035189, -1.522430, 43.263869)
(0.012739, -0.454336, 35.665719)
(-0.010511, 0.430271, 40.934903)
(-0.027399, 0.860928, 31.421874)
(-0.025252, 1.112964, 44.074390)
\end{lstlisting}

\section*{Question 1: Part 2 | Conjugate Gradient method}

Number of directions computed: \( m = 14 \)

\section*{Question 1: Part 3 | Matrix M}

Eigenvalues of M:\@
[1. 1. 1. 1. 1. 1. 1. 1. 1. 1. 1. 1. 1. 1.]

\( M \approx \I_{14 \times 14} \)

\section*{Question 1: Part 4 | Cosine similarity}

List of cosine similarities:
\begin{lstlisting}[firstnumber=0]
0.9999999999999999,
1.0,
1.0,
1.0,
0.9999999999999998,
1.0,
1.0,
1.0000000000000002,
0.9999999999999998,
1.0,
0.9999999999999998,
1.0,
1.0000000000000002,
1.0000000000000002,
\end{lstlisting}

\section*{Question 1: Part 5 | Purpose}
