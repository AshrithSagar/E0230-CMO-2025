\section*{Question 3: Part 1 | Newton's method}

\subsection*{Error norms}

\begin{figure}[h!]
    \centering
    \includegraphics[width=\textwidth]
    {../Codes/figures/Q3-1a-Newton_error_norms_24233.jpeg}% chktex 8
    \caption{
        Error norm \( \norm{\x_k - \xstar}_2 \) (in log scale) vs iterations \( k \) for \texttt{NEWTON\_SOLVE} with different initial points
    }\label{fig:q3.1a-error-norms}
\end{figure}

\newpage
\subsection*{Contour plots}

\begin{figure}[h!]
    \centering
    \begin{subfigure}[b]{0.49\textwidth}
        \includegraphics[width=\textwidth]
        {../Codes/figures/Q3-1b-Newton_path_1_24233.jpeg}% chktex 8
        \caption{
            Initial point
            \(
                \x_0 =
                \begin{bmatrix}
                    2 \\ 2
                \end{bmatrix}
            \)
        }\label{fig:q3.1b-contour-1}
    \end{subfigure}
    \hfill
    \begin{subfigure}[b]{0.49\textwidth}
        \includegraphics[width=\textwidth]
        {../Codes/figures/Q3-1b-Newton_path_2_24233.jpeg}% chktex 8
        \caption{
            Initial point
            \(
                \x_0 =
                \begin{bmatrix}
                    5 \\ 5
                \end{bmatrix}
            \)
        }\label{fig:q3.1b-contour-2}
    \end{subfigure}
    \vskip\baselineskip{}
    \begin{subfigure}[b]{0.49\textwidth}
        \includegraphics[width=\textwidth]
        {../Codes/figures/Q3-1b-Newton_path_3_24233.jpeg}% chktex 8
        \caption{
            Initial point
            \(
                \x_0 =
                \begin{bmatrix}
                    10 \\ -4
                \end{bmatrix}
            \)
        }\label{fig:q3.1b-contour-3}
    \end{subfigure}
    \hfill
    \begin{subfigure}[b]{0.49\textwidth}
        \includegraphics[width=\textwidth]
        {../Codes/figures/Q3-1b-Newton_path_4_24233.jpeg}% chktex 8
        \caption{
            Initial point
            \(
                \x_0 =
                \begin{bmatrix}
                    50 \\ 60
                \end{bmatrix}
            \)
        }\label{fig:q3.1b-contour-4}
    \end{subfigure}
    \caption{
        Contour plots of the rosenbrock function with paths taken by \texttt{NEWTON\_SOLVE} from different initial points clipped in the region \( [-10, 10] \times [-10, 10] \)
    }\label{fig:q3.1b-contours}
\end{figure}

\newpage
\section*{Question 3: Part 2 | Analysis}

As can be seen from the error norms graph~(\autoref{fig:q3.1a-error-norms}), Newton's method converges in very few iterations (4--5) for all initial points tried.

The contour plots~(\autoref{fig:q3.1b-contours}) show that the method takes large steps towards the minimizer \( \xstar \) in each iteration for starting points far away from the minimizer~(\autoref{fig:q3.1b-contour-2},~\ref{fig:q3.1b-contour-3},~\ref{fig:q3.1b-contour-4}).
As in turns out, all of the different initial points tried converge to the same minimizer
\(
    \xstar =
    \begin{bmatrix}
        1 & 1
    \end{bmatrix}^\top
\), which is the unique global minimizer of the rosenbrock function.
