\section*{Question 5}

Consider Quasi Newton Algorithms for minimizing \( f: \R^d \).
Let
\begin{equation*}
    \delta_k = A_{k+1} \gamma_k, \quad \gamma_k = \nabla f\left(\x^{k+1}\right)-\nabla f(\x^k),
    \quad \delta_k = \x^{k+1}-\x^k
\end{equation*}

\( A_{k+1} \in \PSD \)be the Quasi Newton update.
Let \( A_{k+1} = A_k+B_k D B_k^\top \) where \( B_k = \left[\delta_k, A_k \gamma_k\right] \in \R^{d \times 2} \), \( D =
    \begin{bmatrix}
        a & b \\ c & d
    \end{bmatrix}
\in \R^{2 \times 2} D \) is chosen to satisfy the Quasi Newton update.
This is another view of the Broyden family.

\begin{enumerate}[label= (\alph*)]
    \item Determine \( u \in \R^2 \) if \( u = D B_k^\top \gamma_k \).
        State any relation thus obtained on the entries of \( D \) and the quasi newton updates.

    \item What update does the condition \( b = c = 0 \) yield?
        Use the relation derived in the previous question?

    \item Determine \( A_{k+1} \) and \( t \) if \( a = d = t, b = c = -t \) and \( t > 0 \).
        Does this yield any familiar result?

    \item Examine the problem of Computing \( A_{k+1} \) if all entries of \( D \) are same?
\end{enumerate}

\subsection*{Solution}

\subsubsection*{(a)}

\begin{align*}
    \because
    \A_{k+1}
    & =
    \A_k + \B_k \D \B_k^\top
    \implies
    \boldsymbol{\delta}_k
    =
    \A_{k+1} \boldsymbol{\gamma_k}
    =
    \A_k \boldsymbol{\gamma_k}
    + \B_k \D \B_k^\top \boldsymbol{\gamma_k}
    \\
    \implies
    \boldsymbol{\delta}_k - \A_k \boldsymbol{\gamma_k}
    & =
    \B_k \D \B_k^\top \boldsymbol{\gamma_k}
    =
    \B_k \u
    \\
    \text{Also, }
    \boldsymbol{\delta}_k - \A_k \boldsymbol{\gamma_k}
    & =
    \begin{bmatrix}
        \boldsymbol{\delta}_k
        &
        \A_k \boldsymbol{\gamma_k}
    \end{bmatrix}
    \begin{bmatrix}
        1 \\ -1
    \end{bmatrix}
    =
    \B_k
    \begin{bmatrix}
        1 \\ -1
    \end{bmatrix}
    \implies
    \boxed{
        \u
        =
        \begin{bmatrix}
            1 \\ -1
        \end{bmatrix}
    }
\end{align*}

\begin{align*}
    \text{Now, }
    \u
    & =
    \D \B_k^\top \boldsymbol{\gamma_k}
    =
    \begin{bmatrix}
        a & b \\ c & d
    \end{bmatrix}
    \begin{bmatrix}
        \boldsymbol{\delta}_k^\top \boldsymbol{\gamma_k} \\
        \boldsymbol{\gamma_k}^\top \A_k \boldsymbol{\gamma_k}
    \end{bmatrix}
    =
    \begin{bmatrix}
        a \boldsymbol{\delta}_k^\top \boldsymbol{\gamma_k} + b \boldsymbol{\gamma_k}^\top \A_k \boldsymbol{\gamma_k} \\
        c \boldsymbol{\delta}_k^\top \boldsymbol{\gamma_k} + d \boldsymbol{\gamma_k}^\top \A_k \boldsymbol{\gamma_k}
    \end{bmatrix}
    =
    \begin{bmatrix}
        1 \\ -1
    \end{bmatrix}
    \\
    \implies
    &
    \boxed{
        \begin{cases}
            a \boldsymbol{\delta}_k^\top \boldsymbol{\gamma_k} + b \boldsymbol{\gamma_k}^\top \A_k \boldsymbol{\gamma_k} = 1 \\
            c \boldsymbol{\delta}_k^\top \boldsymbol{\gamma_k} + d \boldsymbol{\gamma_k}^\top \A_k \boldsymbol{\gamma_k} = -1
        \end{cases}
    }
\end{align*}

\subsubsection*{(b)}

\begin{align*}
    b = c = 0
    \implies
    &
    \begin{cases}
        a \boldsymbol{\delta}_k^\top \boldsymbol{\gamma_k} = 1 \\
        d \boldsymbol{\gamma_k}^\top \A_k \boldsymbol{\gamma_k} = -1
    \end{cases}
    \implies
    a
    =
    \frac{1}{\boldsymbol{\delta}_k^\top \boldsymbol{\gamma_k}},
    \quad
    d
    =
    -\frac{1}{\boldsymbol{\gamma_k}^\top \A_k \boldsymbol{\gamma_k}}
    \\
    \implies
    \A_{k+1}
    & =
    \A_k
    +
    \begin{bmatrix}
        \boldsymbol{\delta}_k
        &
        \A_k \boldsymbol{\gamma_k}
    \end{bmatrix}
    \begin{bmatrix}
        a & 0 \\ 0 & d
    \end{bmatrix}
    \begin{bmatrix}
        \boldsymbol{\delta}_k^\top \\
        \boldsymbol{\gamma_k}^\top \A_k
    \end{bmatrix}
    =
    \A_k
    +
    a \boldsymbol{\delta}_k \boldsymbol{\delta}_k^\top
    +
    d \A_k \boldsymbol{\gamma_k} \boldsymbol{\gamma_k}^\top \A_k
    \\
    \implies
    \A_{k+1}
    & =
    \A_k
    +
    \frac{\boldsymbol{\delta}_k \boldsymbol{\delta}_k^\top}{\boldsymbol{\delta}_k^\top \boldsymbol{\gamma_k}}
    -
    \frac{\A_k \boldsymbol{\gamma_k} \boldsymbol{\gamma_k}^\top \A_k}{\boldsymbol{\gamma_k}^\top \A_k \boldsymbol{\gamma_k}}
\end{align*}

This is the DFP update.

\subsubsection*{(c)}

\begin{align*}
    &
    a = d = t, b = c = -t, t > 0
    \implies
    \begin{cases}
        t \boldsymbol{\delta}_k^\top \boldsymbol{\gamma_k} - t \boldsymbol{\gamma_k}^\top \A_k \boldsymbol{\gamma_k} = 1 \\
        -t \boldsymbol{\delta}_k^\top \boldsymbol{\gamma_k} + t \boldsymbol{\gamma_k}^\top \A_k \boldsymbol{\gamma_k} = -1
    \end{cases}
    \\
    \implies
    t
    & =
    \frac{1}{\boldsymbol{\delta}_k^\top \boldsymbol{\gamma_k} - \boldsymbol{\gamma_k}^\top \A_k \boldsymbol{\gamma_k}}
    \\
    \implies
    \A_{k+1}
    & =
    \A_k
    +
    \begin{bmatrix}
        \boldsymbol{\delta}_k
        &
        \A_k \boldsymbol{\gamma_k}
    \end{bmatrix}
    \begin{bmatrix}
        t & -t \\ -t & t
    \end{bmatrix}
    \begin{bmatrix}
        \boldsymbol{\delta}_k^\top \\
        \boldsymbol{\gamma_k}^\top \A_k
    \end{bmatrix}
    \\ & =
    \A_k
    +
    t
    \left(
        \boldsymbol{\delta}_k - \A_k \boldsymbol{\gamma_k}
    \right)
    {\left(
            \boldsymbol{\delta}_k - \A_k \boldsymbol{\gamma_k}
    \right)}^\top
    \\
    \implies
    \A_{k+1}
    & =
    \A_k
    +
    \frac{
        \left(
            \boldsymbol{\delta}_k - \A_k \boldsymbol{\gamma_k}
        \right)
        {\left(
                \boldsymbol{\delta}_k - \A_k \boldsymbol{\gamma_k}
        \right)}^\top
    }{
        \boldsymbol{\delta}_k^\top \boldsymbol{\gamma_k} - \boldsymbol{\gamma_k}^\top \A_k \boldsymbol{\gamma_k}
    }
\end{align*}

This is the SR1 update, with \( t > 0 \), i.e., the denominator is positive.

\subsubsection*{(d)}

This is equivalent to putting \( t = -t \), i.e., \( t = 0 \), in the previous part, i.e., in the SR1 update, causing the denominator to blow up.
In this case, the update is not defined.

Another way to see this, with \( t = a = b = c = d \), the system of equations becomes
\begin{equation*}
    \begin{cases}
        t \boldsymbol{\delta}_k^\top \boldsymbol{\gamma_k} + t \boldsymbol{\gamma_k}^\top \A_k \boldsymbol{\gamma_k} = 1 \\
        t \boldsymbol{\delta}_k^\top \boldsymbol{\gamma_k} + t \boldsymbol{\gamma_k}^\top \A_k \boldsymbol{\gamma_k} = -1
    \end{cases}
\end{equation*}
which is an inconsistent system.
The update is not defined in this case.
