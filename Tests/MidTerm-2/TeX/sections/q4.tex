\section*{Question 4}

Consider Newton method for minimizing \( \mathcal{C}^2 \) function over \( \R^d \).

\begin{enumerate}[label= (\alph*)]
    \item Compute the decrease in function value after one iteration of Newton Method from a point \( \hat{\x} \).

    \item If \( \xstar \) be a local minimum.
        The minimum eigenvalue of Hessian at \( \xstar \) is 1.
        It is known that all \( \x \in\left\{\x \mid\left\|\x-\xstar\right\| \leq 5\right\} \) the Hessian at \( \x \) is positive definite.
        Can you estimate from what starting point around \( \xstar \) we can see the speed up due to Newton Method.

    \item Consider the following problem
        \begin{equation*}
            \min _{\x \in \R^2} f(\x)\left( = x_1^2 + {(2 - x_2)}^3\right)
        \end{equation*}
        Let us solve it through Newton method

        You are given an implementation of Newton's iteration.
        You are however not sure about the correctness of the implementation and you suspect there maybe bugs in the code.
        After applying 4 iterations, starting from \( {[1,1]}^\top \) it was found \( \x^4 = {[1,2]}^\top \).
        Is the implementation buggy?
        Give reasons.
\end{enumerate}

\subsection*{Solution}
