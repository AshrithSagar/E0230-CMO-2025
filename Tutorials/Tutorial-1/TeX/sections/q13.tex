\section*{Question 13}

Let \( \phi(x, y) \) be any non-negative function defined on the unit circle \( \left\{(x, y): x^{2}+y^{2}=1\right\} \).
Define
\begin{equation*}
    f(x, y)=
    \begin{cases}
        0, & \text { if } x^{2}+y^{2}<1 \\
        \phi(x, y), & \text { if } x^{2}+y^{2}=1
    \end{cases}
\end{equation*}
\begin{enumerate}[label= (\alph*), noitemsep]
    \item Show that \( f \) is convex.

    \item Determine the range of \( f \), i.e., the set \( \left\{f(x, y): x^{2}+y^{2} \leq 1\right\} \)

    \item Under what conditions is the range of \( f \)
        (i) closed,
        (ii) convex?
\end{enumerate}

\subsection*{Solution}

\subsubsection*{(a) \( f \) is convex}

To show that \( f \) is convex, we need to verify that for any two points \( (x_1, y_1) \) and \( (x_2, y_2) \) in the domain of \( f \), and for any \( \theta \in [0, 1] \), the following inequality holds:
\begin{equation*}
    f(\theta (x_1, y_1) + (1 - \theta) (x_2, y_2)) \leq \theta f(x_1, y_1) + (1 - \theta) f(x_2, y_2)
\end{equation*}

We consider three cases based on the locations of the points \( (x_1, y_1) \) and \( (x_2, y_2) \):
\begin{itemize}
    \item \textbf{Case 1: Both points are inside the unit circle.}

        If both \( (x_1, y_1) \) and \( (x_2, y_2) \) are inside the unit circle, then \( f(x_1, y_1) = 0 \) and \( f(x_2, y_2) = 0 \).
        The convex combination \( \theta (x_1, y_1) + (1 - \theta) (x_2, y_2) \) will also lie inside the unit circle.
        Therefore,
        \begin{equation*}
            f(\theta (x_1, y_1) + (1 - \theta) (x_2, y_2)) = 0
        \end{equation*}
        and
        \begin{equation*}
            \theta f(x_1, y_1) + (1 - \theta) f(x_2, y_2) = 0
        \end{equation*}
        Thus, the inequality holds.

    \item \textbf{Case 2: One point is on the boundary and the other is inside the unit circle.}

        Without loss of generality, let \( (x_1, y_1) \) be on the boundary and \( (x_2, y_2) \) be inside the unit circle.
        Then \( f(x_1, y_1) = \phi(x_1, y_1) \geq 0 \) and \( f(x_2, y_2) = 0 \).
        The convex combination \( \theta (x_1, y_1) + (1 - \theta) (x_2, y_2) \) will lie inside or on the boundary of the unit circle.
        If it lies inside, then
        \begin{equation*}
            f(\theta (x_1, y_1) + (1 - \theta) (x_2, y_2)) = 0
        \end{equation*}
        and
        \begin{equation*}
            \theta f(x_1, y_1) + (1 - \theta) f(x_2, y_2) = \theta \phi(x_1, y_1) \geq 0
        \end{equation*}
        Thus, the inequality holds.
        If it lies on the boundary, then
        \begin{equation*}
            f(\theta (x_1, y_1) + (1 - \theta) (x_2, y_2)) = \phi(\theta (x_1, y_1) + (1 - \theta) (x_2, y_2))
        \end{equation*}
        and since \( \phi \) is non-negative, the inequality still holds.

    \item \textbf{Case 3: Both points are on the boundary of the unit circle.}

        If both \( (x_1, y_1) \) and \( (x_2, y_2) \) are on the boundary, then \( f(x_1, y_1) = \phi(x_1, y_1) \geq 0 \) and \( f(x_2, y_2) = \phi(x_2, y_2) \geq 0 \).
        The convex combination \( \theta (x_1, y_1) + (1 - \theta) (x_2, y_2) \) will lie inside or on the boundary of the unit circle.
        If it lies inside, then
        \begin{equation*}
            f(\theta (x_1, y_1) + (1 - \theta) (x_2, y_2)) = 0
        \end{equation*}
        and
        \begin{equation*}
            \theta f(x_1, y_1) + (1 - \theta) f(x_2, y_2) = \theta \phi(x_1, y_1) + (1 - \theta) \phi(x_2, y_2) \geq 0
        \end{equation*}
        Thus, the inequality holds.
        If it lies on the boundary, then
        \begin{equation*}
            f(\theta (x_1, y_1) + (1 - \theta) (x_2, y_2)) = \phi(\theta (x_1, y_1) + (1 - \theta) (x_2, y_2))
        \end{equation*}
        and since \( \phi \) is non-negative and convex on the boundary of the unit circle, the inequality still holds.
\end{itemize}

In all cases, the convexity condition is satisfied, hence \( f \) is convex.
