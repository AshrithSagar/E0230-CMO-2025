\section*{Question 10}

Suppose \( A, B \) are symmetric and that the problems
\begin{equation*}
    \text{(P1)} \quad \argmin_{x} x^{\top} A x \quad \text { and } \quad (\text{P2}) \quad \argmin_{x} x^{\top} B x
\end{equation*}
have unique solutions \( x_{P1} = x_{P2} = 0 \).
What is the solution to
\begin{equation*}
    \argmin_{x} x^{\top} A B x,
\end{equation*}
and is it unique?
(hint: every symmetric PD matrix has a unique, positive definite square root --- can you prove this?)

\subsection*{Solution}

Let \( \mathbf{A}, \mathbf{B} \in \mathbb{R}^{n \times n} \) be symmetric matrices such that
\begin{equation*}
    \mathbf{x}^*_{\mathbf{A}}
    =
    \argmin_{\mathbf{x} \in \mathbb{R}^n} \mathbf{x}^{\top} \mathbf{A} \mathbf{x}
    =
    \mathbf{0}
    =
    \mathbf{x}^*_{\mathbf{B}}
    =
    \argmin_{\mathbf{x} \in \mathbb{R}^n} \mathbf{x}^{\top} \mathbf{B} \mathbf{x}
\end{equation*}
and that \( \mathbf{x}^*_{\mathbf{A}} \) and \( \mathbf{x}^*_{\mathbf{B}} \) are unique.
Since \( \mathbf{x}^*_{\mathbf{A}} \) and \( \mathbf{x}^*_{\mathbf{B}} \) minimise the quadratic forms \( \mathbf{x}^{\top} \mathbf{A} \mathbf{x} \) and \( \mathbf{x}^{\top} \mathbf{B} \mathbf{x} \) respectively, we have that \( \mathbf{x}^{\top} \mathbf{A} \mathbf{x} > {\mathbf{x}^*_{\mathbf{A}}}^{\top} \mathbf{A} \mathbf{x}^*_{\mathbf{A}} = 0 \) and \( \mathbf{x}^{\top} \mathbf{B} \mathbf{x} > {\mathbf{x}^*_{\mathbf{B}}}^{\top} \mathbf{B} \mathbf{x}^*_{\mathbf{B}} = 0 \) for all \( \mathbf{x} \neq \mathbf{0} \), which means that both \( \mathbf{A} \) and \( \mathbf{B} \) are positive definite (PD) matrices.
Now, consider the problem
\begin{equation*}
    \mathbf{x}^*_{\mathbf{A} \mathbf{B}}
    =
    \argmin_{\mathbf{x} \in \mathbb{R}^n} \mathbf{x}^{\top} \mathbf{A} \mathbf{B} \mathbf{x}
\end{equation*}
Let \( \mathbf{B}^{1/2} \) be the unique positive definite square root of \( \mathbf{B} \), which exists since \( \mathbf{B} \) is symmetric positive definite, and let \( \mathbf{y} = \mathbf{B}^{1/2} \mathbf{x} \implies \mathbf{x} = \mathbf{B}^{-1/2} \mathbf{y} \), then we have
\begin{equation*}
    \mathbf{x}^{\top} \mathbf{A} \mathbf{B} \mathbf{x}
    =
    {\left( \mathbf{B}^{-1/2} \mathbf{y} \right)}^{\top} \mathbf{A} \mathbf{B} \left( \mathbf{B}^{-1/2} \mathbf{y} \right)
    =
    \mathbf{y}^{\top} {\left( \mathbf{B}^{-1/2} \right)}^{\top} \mathbf{A} \mathbf{B} \mathbf{B}^{-1/2} \mathbf{y}
    =
    \mathbf{y}^{\top} \mathbf{B}^{-1/2} \mathbf{A} \mathbf{B}^{1/2} \mathbf{y}
\end{equation*}

Since \( \mathbf{B}^{-1/2} \mathbf{A} \mathbf{B}^{1/2} \sim \mathbf{A} \) (similar matrices have the same eigenvalues), we have that \( \mathbf{B}^{-1/2} \mathbf{A} \mathbf{B}^{1/2} \) is also positive definite, and hence the problem
\begin{equation*}
    \argmin_{\mathbf{y} \in \mathbb{R}^n} \mathbf{y}^{\top} \mathbf{B}^{-1/2} \mathbf{A} \mathbf{B}^{1/2} \mathbf{y}
\end{equation*}
has a unique solution \( \mathbf{y}^* = \mathbf{0} \).
Thus, the solution to the original problem is
\begin{equation*}
    \mathbf{x}^*_{\mathbf{A} \mathbf{B}}
    =
    \mathbf{B}^{-1/2} \mathbf{y}^*
    =
    \mathbf{B}^{-1/2} \mathbf{0}
    =
    \mathbf{0}
\end{equation*}
and it is unique.

\begin{lemma*}
    Every symmetric positive definite matrix has a unique positive definite square root.
\end{lemma*}

\begin{proof}
    Let \( \mathbf{M} \in \mathbb{R}^{n \times n} \) be a symmetric positive definite matrix.
    Since \( \mathbf{M} \) is symmetric, we have that \( \mathbf{M} \) is diagonalizable from the spectral decomposition theorem, i.e., there exists an orthogonal matrix \( \mathbf{Q} \) and a diagonal matrix \( \mathbf{D} \) such that
    \begin{equation*}
        \mathbf{M} = \mathbf{Q} \mathbf{D} \mathbf{Q}^{\top}
    \end{equation*}
    where the diagonal entries of \( \mathbf{D} \) are the eigenvalues of \( \mathbf{M} \).
    Since \( \mathbf{M} \) is positive definite, all its eigenvalues are positive, i.e., \( \lambda_i > 0 \) for all \( i = 1, 2, \ldots, n \).
    Now, we can define the square root of \( \mathbf{M} \) as
    \begin{equation*}
        \mathbf{M}^{1/2} = \mathbf{Q} \mathbf{D}^{1/2} \mathbf{Q}^{\top}
    \end{equation*}
    where \( \mathbf{D}^{1/2} \) is the diagonal matrix with diagonal entries \( \sqrt{\lambda_i} \) for all \( i = 1, 2, \ldots, n \).
    It is easy to verify that \( {(\mathbf{M}^{1/2})}^2 = \mathbf{M} \) since
    \begin{equation*}
        \left( \mathbf{Q} \mathbf{D}^{1/2} \mathbf{Q}^{\top} \right) \left( \mathbf{Q} \mathbf{D}^{1/2} \mathbf{Q}^{\top} \right)
        =
        \mathbf{Q} \mathbf{D}^{1/2} \left( \cancelto{\mathbf{I}}{ \mathbf{Q}^{\top} \mathbf{Q} } \right) \mathbf{D}^{1/2} \mathbf{Q}^{\top}
        =
        \mathbf{Q} \left( \mathbf{D}^{1/2} \mathbf{D}^{1/2} \right) \mathbf{Q}^{\top}
        =
        \mathbf{Q} \mathbf{D} \mathbf{Q}^{\top}
        =
        \mathbf{M}
    \end{equation*}
    It is easy to verify that \( \mathbf{M}^{1/2} \) is symmetric and positive definite, since \( \sqrt{\lambda_i} > 0 \) for all \( i = 1, 2, \ldots, n \).

    To show the uniqueness of the positive definite square root, suppose there exists another symmetric positive definite matrix \( \mathbf{N} \) such that \( \mathbf{N}^2 = \mathbf{M} \).
    Then, we have
    \begin{equation*}
        \mathbf{N}^2 = \mathbf{M} = \mathbf{Q} \mathbf{D} \mathbf{Q}^{\top}
    \end{equation*}
    Since \( \mathbf{N} \) is symmetric, it is also diagonalizable, i.e., there exists an orthogonal matrix \( \mathbf{P} \) and a diagonal matrix \( \mathbf{E} \) such that
    \begin{equation*}
        \mathbf{N} = \mathbf{P} \mathbf{E} \mathbf{P}^{\top}
    \end{equation*}
    where the diagonal entries of \( \mathbf{E} \) are the eigenvalues of \( \mathbf{N} \).
    Since \( \mathbf{N} \) is positive definite, all its eigenvalues are positive, i.e., \( \mu_i > 0 \) for all \( i = 1, 2, \ldots, n \).
    Now, we have
    \begin{equation*}
        \mathbf{N}^2 = \mathbf{P} \mathbf{E}^2 \mathbf{P}^{\top} = \mathbf{Q} \mathbf{D} \mathbf{Q}^{\top}
    \end{equation*}
    This implies that \( \mathbf{E}^2 = \mathbf{D} \), since both sides are diagonal matrices with the same eigenvalues.
    Thus, we have \( \mu_i^2 = \lambda_i \) for all \( i = 1, 2, \ldots, n \).
    Since \( \mu_i > 0 \) and \( \lambda_i > 0 \), we have \( \mu_i = \sqrt{\lambda_i} \) for all \( i = 1, 2, \ldots, n \).
    Thus, we have \( \mathbf{E} = \mathbf{D}^{1/2} \) and hence \( \mathbf{N} = \mathbf{P} \mathbf{D}^{1/2} \mathbf{P}^{\top} \).
    Now, since \( \mathbf{N}^2 = \mathbf{M} \), we have
    \begin{equation*}
        \mathbf{P} \mathbf{D}^{1/2} \mathbf{P}^{\top} \mathbf{P} \mathbf{D}^{1/2} \mathbf{P}^{\top}
        =
        \mathbf{Q} \mathbf{D} \mathbf{Q}^{\top}
        \implies
        \mathbf{P} \mathbf{D} \mathbf{P}^{\top}
        =
        \mathbf{Q} \mathbf{D} \mathbf{Q}^{\top}
    \end{equation*}
    This implies that \( \mathbf{P} = \mathbf{Q} \) and hence \( \mathbf{N} = \mathbf{M}^{1/2} \).
    Thus, the positive definite square root of a symmetric positive definite matrix is unique.
\end{proof}
