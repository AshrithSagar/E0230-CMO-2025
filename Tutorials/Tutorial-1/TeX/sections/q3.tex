\section*{Question 3}

Suppose \( x \in \mathbb{R} \).
Show that \( e^{x} \geq 1+x \).

\subsection*{Solution}

We have
\begin{align*}
  e^{x}
  & =
  \sum_{n = 0}^{\infty} \frac{x^{n}}{n!} = 1 + x + \sum_{n=2}^{\infty} \frac{x^{n}}{n!}
\end{align*}

\paragraph{Alternate solution:}

Define
\begin{equation*}
  g(x) \triangleq e^{x} - (1+x).
\end{equation*}
Then we have
\begin{equation*}
  g'(x) = e^{x} - 1, \quad g''(x) = e^{x}.
\end{equation*}

Note that \( g''(x) > 0 \) for all \( x \in \mathbb{R} \), which implies that \( g'(x) \) is strictly increasing.

Also, \( g'(0) = 0 \).
Thus, \( g'(x) < 0 \) for all \( x < 0 \) and \( g'(x) > 0 \) for all \( x > 0 \).
This implies that \( g(x) \) is strictly decreasing for all \( x < 0 \) and strictly increasing for all \( x > 0 \).
