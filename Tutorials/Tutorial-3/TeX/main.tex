\documentclass[12pt]{article}

\usepackage[utf8]{inputenc}
\usepackage{amsfonts, amsmath, amssymb, amsthm, cancel, interval, latexsym, mathcommand, mathtools}
\usepackage{enumitem, fancybox, geometry, graphicx, hyperref, microtype, titlesec, tocloft, xparse}
\usepackage[linesnumbered,ruled,vlined]{algorithm2e}
\usepackage[most]{tcolorbox}
\usepackage[numbers,sort]{natbib}
\usepackage[dvipsnames]{xcolor}

\hypersetup{
    colorlinks=true,
    linkcolor=,
    urlcolor=,
}
\geometry{
    left=1in,
    right=1in,
    top=1in,
    bottom=1in
}

\setlength{\parskip}{0pt}
\setlength{\parindent}{0pt}

%% macros.tex

\newcommand{\R}{{\mathbb{R}}}
\newcommand{\N}{{\mathbb{N}}}
\newcommand{\CC}{{\mathbb{C}}}

\newcommand{\notimplies}{\mathrel{{\ooalign{\hidewidth\(\not{\phantom{=}}\)\hidewidth\cr\(\implies\)}}}}

\newcommand{\set}[1]{{\left\{#1\right\}}}
\newcommand{\given}{\;\middle|\;}
\newcommand\spanset[1]{{\mathrm{span}\set{#1}}}

\NewDocumentCommand{\pbrac}{om}{{\IfNoValueTF{#1}{\left\lparen#2\right\rparen}{#1\lparen#2#1\rparen}}}
\NewDocumentCommand{\bbrac}{om}{{\IfNoValueTF{#1}{\left\lbrack#2\right\rbrack}{#1\lbrack#2#1\rbrack}}}

\NewDocumentCommand{\abs}{om}{{\IfNoValueTF{#1}{\left\lvert#2\right\rvert}{#1\lvert#2#1\rvert}}}
\NewDocumentCommand{\norm}{om}{{\IfNoValueTF{#1}{\left\lVert#2\right\rVert}{#1\lVert#2#1\rVert}}}

\newcommand{\dotp}[2]{{{#1}^\top#2}}
\newcommand{\outp}[2]{{#1{#2}^\top}}
\ExplSyntaxOn{}
\NewDocumentCommand{\qf}{mmo}{
    \IfValueTF{#3}{
        \str_if_eq:nnT {#1} {#3} {
            \PackageWarning{macros}{Argument 3 not needed}
        }
        {#1}^\top#2\,#3
    }{
        {#1}^\top#2\,#1
    }
}
\ExplSyntaxOff{}

\newcommand{\inv}[1]{{{#1}^{-1}}}
\newcommand{\trace}[1]{\func{\mathrm{trace}}{#1}}
\newcommand{\determinant}[1]{\func{\mathrm{det}}{#1}}
\newcommand{\rank}[1]{\func{\mathrm{rank}}{#1}}
\newcommand{\diag}[1]{\func{\mathrm{diag}}{#1}}

\newcommand{\func}[2]{{#1\!\left\lparen#2\right\rparen}}
\newcommand{\grad}[2]{{\nabla\!\func{#1}{#2}}}
\newcommand{\hess}[2]{{\nabla^2\!\func{#1}{#2}}}
\newcommand{\hessinv}[2]{\inv{\left[\hess{#1}{#2}\right]}}

\newcommand{\SD}[1][d]{{\calS_{#1}}}
\newcommand{\PSD}[1][d]{{\calS^{+}_{#1}}}
\newcommand{\PD}[1][d]{{\calS^{++}_{#1}}}

\LoopCommands{CDSOK}[cal#1]{\newmathcommand#2{{\mathcal{#1}}}}
\newcommand{\classC}{{\calC}}

\LoopCommands{{argmin}{argmax}}[#1]{\newmathcommand#2{\mathop{\mathrm{#1}}}}
\LoopCommands{{minimize}{maximize}}[#1]{\DeclareMathOperator*{#2}{#1}}
\DeclareMathOperator{\subjectto}{subject~to}
\newcommand{\smalloh}[1]{\func{{\scriptstyle\calO}}{#1}}

\LoopCommands{xyzheapbruv}[#1]{\declaremathcommandPIE#2{{\mathbf{#1}}##1##2\IfEmptyTF{##3}{}{^{(\GetExponent{##3})}}}}

\LoopCommands{IUVMCABQED}[#1]{\newmathcommand#2{{\mathbf{#1}}}}
\newcommand{\Qinv}{\inv{\Q}}

\newcommand{\xstar}{{{\mathbf{x}}^\ast}}
\newcommand{\xhat}{{\hat{\x}}}
\newcommand{\ppow}[1]{^{(#1)}}

\newcommand{\half}{{\frac{1}{2}}}
\newcommand{\zero}{{\mathbf{0}}}


\begin{document}

\begin{center}
    \doublebox{
        \begin{minipage}{0.95\textwidth}
            \centering
            \textbf{Course:} E0 230: Computational Methods of Optimisation\\
            \textbf{Tutorial:} Tutorial-3\\
            \textbf{Student Name:} Ashrith Sagar Yedlapalli\\
        \end{minipage}
    }
\end{center}

\newtheorem{lemma}{Lemma}
\newtheorem*{lemma*}{Lemma}
\newtheorem{definition}{Definition}
\newtheorem{theorem}{Theorem}

\renewcommand{\baselinestretch}{1.5}

\section*{Part-I:\@ Conjugate gradient}
\section*{Question 1: CG vs Gradient Descent}

Consider the system
\begin{equation*}
    A =
    \begin{bmatrix}
        2 & 0 \\
        0 & 8
    \end{bmatrix},
    \quad
    b =
    \begin{bmatrix}
        2 \\
        2
    \end{bmatrix}.
\end{equation*}

\begin{enumerate}
    \item Write the equivalent quadratic objective
        \begin{equation*}
            f(x) = \frac{1}{2} x^{\top} A x - b^{\top} x
        \end{equation*}

    \item Starting from \( x_{0} = {(0,0)}^{\top} \), compute:
        \begin{itemize}
            \item One step of steepest descent (take a step along \( -\nabla f\left(x_{0}\right) \) with exact line search).
            \item One step of conjugate gradient.
        \end{itemize}

    \item Compare the new iterates \( x_{1}^{\mathrm{GD}} \) and \( x_{1}^{\mathrm{CG}} \).

    \item Why does gradient descent exhibit a ``zig-zag'' pattern here, while CG terminates in at most 2 steps?
\end{enumerate}

\subsection*{Solution}
\clearpage
\section*{Question 2: Role of the Initial Residual}

Consider
\begin{equation*}
    A =
    \begin{bmatrix}
        4 & 1 \\
        1 & 3
    \end{bmatrix},
    \quad
    b =
    \begin{bmatrix}
        1 \\
        2
    \end{bmatrix}.
\end{equation*}

\begin{enumerate}
    \item Run CG with starting guess \( x_{0} = {(0,0)}^{\top} \).
        Compute the initial residual \( r_{0} \) and the first search direction \( p_{0} \).

    \item Now instead start from \( x_{0} = {(10,10)}^{\top} \).
        Compute the new residual \( r_{0} \) and search direction \( p_{0} \).

    \item Why do the directions change even though the minimiser is the same?
        What does this tell you about the dependence of CG on the initial guess?
\end{enumerate}

\subsection*{Solution}

\subsubsection*{(1)
    \(
        \x_0
        =
        \begin{bmatrix}
            0 &
            0
        \end{bmatrix}^\top
    \)
}

\begin{align*}
    \implies
    \r_0
    & =
    \A \x_0 - \b
    =
    -\b
    =
    \begin{bmatrix}
        4 & 1 \\
        1 & 3
    \end{bmatrix}
    \begin{bmatrix}
        0 \\
        0
    \end{bmatrix}
    -
    \begin{bmatrix}
        1 \\
        2
    \end{bmatrix}
    =
    \begin{bmatrix}
        -1 \\
        -2
    \end{bmatrix}
    \\
    \implies
    \p_0
    & =
    -\r_0
    =
    \begin{bmatrix}
        1 \\
        2
    \end{bmatrix}
\end{align*}

\subsubsection*{(2)
    \(
        \x_0
        =
        \begin{bmatrix}
            10 &
            10
        \end{bmatrix}^\top
    \)
}

\begin{align*}
    \implies
    \r_0
    & =
    \A \x_0 - \b
    =
    \begin{bmatrix}
        4 & 1 \\
        1 & 3
    \end{bmatrix}
    \begin{bmatrix}
        10 \\
        10
    \end{bmatrix}
    -
    \begin{bmatrix}
        1 \\
        2
    \end{bmatrix}
    =
    \begin{bmatrix}
        50 \\
        40
    \end{bmatrix}
    -
    \begin{bmatrix}
        1 \\
        2
    \end{bmatrix}
    =
    \begin{bmatrix}
        49 \\
        38
    \end{bmatrix}
    \\
    \implies
    \p_0
    & =
    -\r_0
    =
    \begin{bmatrix}
        -49 \\
        -38
    \end{bmatrix}
\end{align*}

\subsubsection*{(3) Role of inital residual}

The search directions upto iteration \( k \) (inclusive) lie in the Krylov subspace \( \func{\mathcal{K}_k}{\A, \r_0} \), i.e.,
\begin{equation*}
    \func{\mathcal{K}_k}{\A, \r_0}
    =
    \spanset{
        \r_0,
        \A \r_0,
        \A^2 \, \r_0,
        \ldots,
        \A^{k-1} \, \r_0
    }
    =
    \spanset{\A^i \, \r_0}_{i = 0}^{k-1}
\end{equation*}
and the CG procedure uptill iteration \( k \) is equivalent to minimizing the quadratic function \( f \) over the affine space \( \x_0 + \func{\mathcal{K}_k}{\A, \r_0} \).
At \( k = d \), since \( \spanset{\p_i}_{i = 0}^{k - 1} \) are linearly independent, we get that the Krylov subspace becomes \( \R^d \) for any \( r_0 \), thereby, the final iterate \( \x^d \) is independent of the initial residual \( \r_0 \), but the intermediate search directions and iterates depend on \( \r_0 \).
\clearpage
\section*{Question 3: Error in the Energy Norm}

Take
\begin{equation*}
    A =
    \begin{bmatrix}
        6 & 2 \\
        2 & 3
    \end{bmatrix},
    \quad
    b =
    \begin{bmatrix}
        2 \\
        1
    \end{bmatrix}.
\end{equation*}

\begin{enumerate}
    \item Compute the exact solution \( x^{*} = A^{-1} b \).

    \item Run one step of CG from \( x_{0} = {(0,0)}^{\top} \), giving \( x_{1} \).

    \item Compute the error \( e_{1} = x^{*}-x_{1} \).

    \item Verify that \( e_{1} \) is orthogonal to the Krylov subspace in the \( A \)-inner product, i.e.
        \begin{equation*}
            p_{0}^{\top} A e_{1} = 0
        \end{equation*}
        What does this mean geometrically?
\end{enumerate}

\subsection*{Solution}
\clearpage
\section*{Part-II:\@ Newton's method}
\section*{Question 1: Local vs Global Convergence}

Consider the function
\begin{equation*}
    f(x) = x^{3}-2 x+2
\end{equation*}

\begin{enumerate}
    \item Write down \( f^{\prime}(x) \) and \( f^{\prime \prime}(x) \).

    \item Apply Newton's method starting from \( x_{0} = 0 \) and compute the first two iterates.

    \item Repeat starting from \( x_{0} = -2 \).

    \item Compare the two behaviours.
        Why does Newton's method converge rapidly in one case but not in the other?
        What does this illustrate about the importance of the starting point?
\end{enumerate}

\subsection*{Solution}
\clearpage
\section*{Question 2: Quadratic Convergence Inside the Newton Region}

Consider
\begin{equation*}
    f(x) = x^{2} + e^{x}
\end{equation*}

\begin{enumerate}
    \item Write down \( f^{\prime}(x) \) and \( f^{\prime \prime}(x) \).

    \item Starting from \( x_{0} = 0 \), compute two steps of Newton's method to approximate the minimiser.

    \item Compute the error \( \left|x_{k}-x^{*}\right| \) at each step (where \( x^{*} \) is the exact minimiser, approximate it numerically).

    \item Do you observe quadratic convergence?
        Why does this happen once the iterates are sufficiently close to \( x^{*} \)?
\end{enumerate}

\subsection*{Solution}
\clearpage
\section*{Question 3: When Newton Fails}

Consider the function

\begin{equation*}
    f(x) = |x| + x^{2}
\end{equation*}

\begin{enumerate}
    \item Explain why Newton's method cannot be directly applied at \( x = 0 \).

    \item Starting from \( x_{0} = 1 \), compute one step of Newton's method (using \( f^{\prime}(x) \) and \( f^{\prime \prime}(x) \) wherever they exist).

    \item What happens if you start from \( x_{0} = -1 \) instead?

    \item What does this example illustrate about the limitations of Newton's method when the function is not twice differentiable?
\end{enumerate}

\subsection*{Solution}

\vspace{-1em}
\begin{equation*}
    \implies
    f(x) =
    \begin{cases}
        x^2 + x, & x \geq 0 \\
        x^2 - x, & x < 0
    \end{cases}
\end{equation*}

\vspace{-1em}
\subsection*{(1) Newton's method at \( x = 0 \)}

\vspace{-1em}
\begin{align*}
    \implies
    f'(x)
    & =
    \begin{cases}
        2x + 1, & x > 0 \\
        2x - 1, & x < 0
    \end{cases}
    \\
    f''(x)
    & =
    2,
    \quad x \neq 0
\end{align*}

At \( x = 0 \), \( f'(x) \) is \underline{not defined}, hence Newton's method cannot be directly applied at \( x = 0 \).

\vspace{-1em}
\subsection*{(2) Newton's method, \( x_0 = 1 \)}

\vspace{-1em}
\begin{align*}
    \implies
    x_1
    & =
    x_0 - \frac{f'(x_0)}{f''(x_0)}
    =
    1 - \frac{3}{2}
    \implies
    \boxed{
        x_1
        =
        -\half
    }
    < 0
\end{align*}

\vspace{-1em}
\subsection*{(3) Newton's method, \( x_0 = -1 \)}

\vspace{-1em}
\begin{align*}
    \implies
    x_1
    & =
    x_0 - \frac{f'(x_0)}{f''(x_0)}
    =
    -1 - \frac{-3}{2}
    \implies
    \boxed{
        x_1
        =
        \half
    }
    > 0
\end{align*}

\vspace{-1em}
\subsection*{(4) Limitations of Newton's method}

This example exhibits \underline{oscillatory behavior} when applying Newton's method starting from different initial points.
As can be seen from the iterates, starting with \( x_0 = 1 \) leads to \( x_1 = -\half \), and starting with \( x_0 = -1 \) leads to \( x_1 = \half \).
This oscillation occurs because the function \( f(x) \) is not twice differentiable at \( x = 0 \), where the derivative \( f'(x) \) is discontinuous.
\clearpage

\end{document}
