\section*{Question 2: Quadratic Convergence Inside the Newton Region}

Consider
\begin{equation*}
    f(x) = x^{2} + e^{x}
\end{equation*}

\begin{enumerate}
    \item Write down \( f^{\prime}(x) \) and \( f^{\prime \prime}(x) \).

    \item Starting from \( x_{0} = 0 \), compute two steps of Newton's method to approximate the minimiser.

    \item Compute the error \( \left|x_{k}-x^{*}\right| \) at each step (where \( x^{*} \) is the exact minimiser, approximate it numerically).

    \item Do you observe quadratic convergence?
        Why does this happen once the iterates are sufficiently close to \( x^{*} \)?
\end{enumerate}

\subsection*{Solution}

\subsection*{(1) \( f'(x), f''(x) \)}

\begin{align*}
    \implies
    f'(x)
    & =
    2x + e^x
    \\
    f''(x)
    & =
    2 + e^x
\end{align*}

\subsection*{(2) Newton's method, \( x_0 = 0 \)}

\begin{align*}
    \implies
    x_1
    & =
    0 - \frac{f'(0)}{f''(0)}
    \implies
    \boxed{
        x_1
        =
        -\frac{1}{3}
        \approx
        -0.33333
    }
    \\
    \implies
    x_2
    & =
    -\frac{1}{3}
    - \frac{
        f'(-\frac{1}{3})
    }{
        f''(-\frac{1}{3})
    }
    =
    -\frac{1}{3}
    - \frac{
        -\frac{2}{3} + e^{-\frac{1}{3}}
    }{
        2 + e^{-\frac{1}{3}}
    }
    =
    \frac{
        \cancel{-2} - e^{-\frac{1}{3}} \cancel{+ 2} - 3 e^{-\frac{1}{3}}
    }{
        3(2 + e^{-\frac{1}{3}})
    }
    \\
    \implies
    &
    \boxed{
        x_2
        =
        \frac{
            -4 e^{-\frac{1}{3}}
        }{
            3(2 + e^{-\frac{1}{3}})
        }
        \approx
        -0.35168
    }
\end{align*}

\subsection*{(3) Error \( \abs{x_k - x^\ast} \)}

\begin{align*}
    x^\ast
    & =
    -0.35173,
    \quad
    f(x^\ast)
    =
    0.82718
    \\
    \implies
    \abs{x_0 - x^\ast}
    & =
    \abs{0 - (-0.35173)}
    =
    \boxed{
        0.35173
    }
    \\
    \abs{x_1 - x^\ast}
    & =
    \abs{-0.33333 - (-0.35173)}
    =
    \boxed{
        0.01839
    }
    \\
    \abs{x_2 - x^\ast}
    & =
    \abs{-0.35168 - (-0.35173)}
    =
    \boxed{
        0.00005
    }
\end{align*}

\subsection*{(4) Quadratic convergence}

Yes, we observe quadratic convergence.
This happens because once the iterates are sufficiently close to \( x^\ast \), the Taylor series expansion of \( f \) around \( x^\ast \) is \underline{dominated by the quadratic term}, leading to rapid convergence of Newton's method.

\begin{equation*}
    \frac{0.01839}{0.35173^2}
    \approx
    0.14864
    , \qquad
    \frac{0.00005}{0.01839^2}
    \approx
    0.14784
\end{equation*}
