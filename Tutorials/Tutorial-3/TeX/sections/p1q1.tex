\section*{Question 1: CG vs Gradient Descent}

Consider the system
\begin{equation*}
    A =
    \begin{bmatrix}
        2 & 0 \\
        0 & 8
    \end{bmatrix},
    \quad
    b =
    \begin{bmatrix}
        2 \\
        2
    \end{bmatrix}.
\end{equation*}

\begin{enumerate}
    \item Write the equivalent quadratic objective
        \begin{equation*}
            f(x) = \frac{1}{2} x^{\top} A x - b^{\top} x
        \end{equation*}

    \item Starting from \( x_{0} = {(0,0)}^{\top} \), compute:
        \begin{itemize}
            \item One step of steepest descent (take a step along \( -\nabla f\left(x_{0}\right) \) with exact line search).
            \item One step of conjugate gradient.
        \end{itemize}

    \item Compare the new iterates \( x_{1}^{\mathrm{GD}} \) and \( x_{1}^{\mathrm{CG}} \).

    \item Why does gradient descent exhibit a ``zig-zag'' pattern here, while CG terminates in at most 2 steps?
\end{enumerate}

\subsection*{Solution}

\subsection*{(1) Equivalent quadratic objective}

The equivalent quadratic objective \( f: \R^d \to \R \) is given by
\begin{equation*}
    \func{f}{\x}
    =
    \half \qf{\x}{\A}
    - \dotp{\b}{\x}
\end{equation*}

With
\(
    \x =
    \begin{bmatrix}
        x_1 & x_2
    \end{bmatrix}^\top
\), we get
\begin{equation*}
    \implies
    \func{f}{x_1, x_2}
    =
    \half
    \begin{bmatrix}
        x_1 & x_2
    \end{bmatrix}
    \begin{bmatrix}
        2 & 0 \\
        0 & 8
    \end{bmatrix}
    \begin{bmatrix}
        x_1 \\
        x_2
    \end{bmatrix}
    -
    \begin{bmatrix}
        2 & 2
    \end{bmatrix}
    \begin{bmatrix}
        x_1 \\
        x_2
    \end{bmatrix}
\end{equation*}
\begin{equation*}
    \therefore
    \boxed{
        \func{f}{x_1, x_2}
        =
        x_1^2 + 4 x_2^2 - 2 x_1 - 2 x_2,
        \quad \forall x_1, x_2 \in \R
    }
\end{equation*}
