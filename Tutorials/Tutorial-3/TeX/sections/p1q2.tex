\section*{Question 2: Role of the Initial Residual}

Consider
\begin{equation*}
    A =
    \begin{bmatrix}
        4 & 1 \\
        1 & 3
    \end{bmatrix},
    \quad
    b =
    \begin{bmatrix}
        1 \\
        2
    \end{bmatrix}.
\end{equation*}

\begin{enumerate}
    \item Run CG with starting guess \( x_{0} = {(0,0)}^{\top} \).
        Compute the initial residual \( r_{0} \) and the first search direction \( p_{0} \).

    \item Now instead start from \( x_{0} = {(10,10)}^{\top} \).
        Compute the new residual \( r_{0} \) and search direction \( p_{0} \).

    \item Why do the directions change even though the minimiser is the same?
        What does this tell you about the dependence of CG on the initial guess?
\end{enumerate}

\subsection*{Solution}

\subsubsection*{(1)
    \(
        \x_0
        =
        \begin{bmatrix}
            0 &
            0
        \end{bmatrix}^\top
    \)
}

\begin{align*}
    \implies
    \r_0
    & =
    \A \x_0 - \b
    =
    -\b
    =
    \begin{bmatrix}
        4 & 1 \\
        1 & 3
    \end{bmatrix}
    \begin{bmatrix}
        0 \\
        0
    \end{bmatrix}
    -
    \begin{bmatrix}
        1 \\
        2
    \end{bmatrix}
    =
    \begin{bmatrix}
        -1 \\
        -2
    \end{bmatrix}
    \\
    \implies
    \p_0
    & =
    -\r_0
    =
    \begin{bmatrix}
        1 \\
        2
    \end{bmatrix}
\end{align*}

\subsubsection*{(2)
    \(
        \x_0
        =
        \begin{bmatrix}
            10 &
            10
        \end{bmatrix}^\top
    \)
}

\begin{align*}
    \implies
    \r_0
    & =
    \A \x_0 - \b
    =
    \begin{bmatrix}
        4 & 1 \\
        1 & 3
    \end{bmatrix}
    \begin{bmatrix}
        10 \\
        10
    \end{bmatrix}
    -
    \begin{bmatrix}
        1 \\
        2
    \end{bmatrix}
    =
    \begin{bmatrix}
        50 \\
        40
    \end{bmatrix}
    -
    \begin{bmatrix}
        1 \\
        2
    \end{bmatrix}
    =
    \begin{bmatrix}
        49 \\
        38
    \end{bmatrix}
    \\
    \implies
    \p_0
    & =
    -\r_0
    =
    \begin{bmatrix}
        -49 \\
        -38
    \end{bmatrix}
\end{align*}

\subsubsection*{(3) Role of inital residual}

The search directions upto iteration \( k \) (inclusive) lie in the Krylov subspace \( \func{\mathcal{K}_k}{\A, \r_0} \), i.e.,
\begin{equation*}
    \func{\mathcal{K}_k}{\A, \r_0}
    =
    \spanset{
        \r_0,
        \A \r_0,
        \A^2 \, \r_0,
        \ldots,
        \A^{k-1} \, \r_0
    }
    =
    \spanset{\A^i \, \r_0}_{i = 0}^{k-1}
\end{equation*}
and the CG procedure uptill iteration \( k \) is equivalent to minimizing the quadratic function \( f \) over the affine space \( \x_0 + \func{\mathcal{K}_k}{\A, \r_0} \).
At \( k = d \), since \( \spanset{\p_i}_{i = 0}^{k - 1} \) are linearly independent, we get that the Krylov subspace becomes \( \R^d \) for any \( r_0 \), thereby, the final iterate \( \x^d \) is independent of the initial residual \( \r_0 \), but the intermediate search directions and iterates depend on \( \r_0 \).
