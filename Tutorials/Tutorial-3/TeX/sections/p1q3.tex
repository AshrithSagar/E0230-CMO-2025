\section*{Question 3: Error in the Energy Norm}

Take
\begin{equation*}
    A =
    \begin{bmatrix}
        6 & 2 \\
        2 & 3
    \end{bmatrix},
    \quad
    b =
    \begin{bmatrix}
        2 \\
        1
    \end{bmatrix}.
\end{equation*}

\begin{enumerate}
    \item Compute the exact solution \( x^{*} = A^{-1} b \).

    \item Run one step of CG from \( x_{0} = {(0,0)}^{\top} \), giving \( x_{1} \).

    \item Compute the error \( e_{1} = x^{*}-x_{1} \).

    \item Verify that \( e_{1} \) is orthogonal to the Krylov subspace in the \( A \)-inner product, i.e.
        \begin{equation*}
            p_{0}^{\top} A e_{1} = 0
        \end{equation*}
        What does this mean geometrically?
\end{enumerate}

\subsection*{Solution}

\subsubsection*{(1) Exact solution \( \xstar = \inv{\A} \b \)}

\begin{align*}
    \implies
    \inv{\A}
    & =
    \frac{1}{\det{\A}}
    \begin{bmatrix}
        3 & -2 \\
        -2 & 6
    \end{bmatrix}
    =
    \frac{1}{14}
    \begin{bmatrix}
        3 & -2 \\
        -2 & 6
    \end{bmatrix}
    \\
    \implies
    \xstar
    & =
    \frac{1}{14}
    \begin{bmatrix}
        3 & -2 \\
        -2 & 6
    \end{bmatrix}
    \begin{bmatrix}
        2 \\
        1
    \end{bmatrix}
    =
    \frac{1}{14}
    \begin{bmatrix}
        4 \\
        2
    \end{bmatrix}
    \implies
    \boxed{
        \xstar
        =
        \frac{1}{7}
        \begin{bmatrix}
            2 \\
            1
        \end{bmatrix}
    }
\end{align*}

\subsubsection*{(2) \( \x_1 \)}

\begin{align*}
    \implies
    \r_0
    & =
    \b - \A \x_0
    =
    \b
    \implies
    \p_0
    =
    \r_0
    =
    \b
    \\
    \implies
    \alpha_0
    & =
    \frac{\dotp{\r_0}{\r_0}}{\qf{\p_0}{\A}}
    =
    \frac{5}{
        \begin{bmatrix}
            2 & 1
        \end{bmatrix}
        \begin{bmatrix}
            6 & 2 \\
            2 & 3
        \end{bmatrix}
        \begin{bmatrix}
            2 \\
            1
        \end{bmatrix}
    }
    =
    \frac{5}{
        \begin{bmatrix}
            2 & 1
        \end{bmatrix}
        \begin{bmatrix}
            13 \\
            7
        \end{bmatrix}
    }
    =
    \frac{5}{33}
    \\
    \implies
    \x_1
    & =
    \x_0 + \alpha_0 \p_0
    \implies
    \boxed{
        \x_1
        =
        \frac{5}{33}
        \begin{bmatrix}
            2 \\
            1
        \end{bmatrix}
    }
\end{align*}

\subsubsection*{(3) Error \( \Delta_1 = \xstar - \x_1 \)}

\begin{equation*}
    \Delta_1
    =
    \xstar - \x_1
    =
    \frac{1}{7}
    \begin{bmatrix}
        2 \\
        1
    \end{bmatrix}
    -
    \frac{5}{33}
    \begin{bmatrix}
        2 \\
        1
    \end{bmatrix}
    =
    \boxed{
        \frac{-2}{231}
        \begin{bmatrix}
            2 \\
            1
        \end{bmatrix}
    }
\end{equation*}

\subsubsection*{(4) \( \qf{\p_0}{\A}[\Delta_1] = 0 \)}

This relation would mean that \( \p_0 \) and \( \Delta_1 \) are \( \A \)--conjugate vectors.

\begin{equation*}
    \qf{\p_0}{\A}[\Delta_1]
    =
    \frac{-2}{231}
    \begin{bmatrix}
        2 & 1
    \end{bmatrix}
    \begin{bmatrix}
        6 & 2 \\
        2 & 3
    \end{bmatrix}
    \begin{bmatrix}
        2 \\
        1
    \end{bmatrix}
    =
    \frac{-2}{\cancel{33} \times 7} \cancel{33}
    =
    -\frac{2}{7}
\end{equation*}

The error \( \Delta_k = (\xstar - \x^k) \) is always \( \A \)--orthogonal to the subspace of the gradients computed so far, i.e., \( \spanset{\grad{f}{\x^i}}_{i = 0}^{k-1} \).
