
\part{Unconstrained optimisation}

\chapter{Introduction}

\section{Optimisation problem}

An \textbf{unconstrained minimisation problem} aims to find a point \( \xstar \in \R^d \) that minimises a function \( f: \R^d \to \R \), i.e.,
\vspace{-1.5em}
\begin{align*}
    &
    \minimize_{\x \in \R^d} \func{f}{\x}
    \\
    \func{f}{\xstar}
    =
    \min_{\x \in \R^d} \func{f}{\x}
    \quad \iff \quad
    \xstar
    & =
    \argmin_{\x \in \R^d} \func{f}{\x}
    \quad \iff \quad
    \func{f}{\xstar}
    \leq
    \func{f}{\x},
    \quad \forall \x \in \R^d
\end{align*}

The function \( f \) is called the \textbf{objective function}, the point \( \xstar \) is called the \textbf{minimum point}, and the value \( \func{f}{\xstar} \) is called the \textbf{minimum value}.

\subsection{Examples}

% chktex-file 44
\begin{table}[h!]
    \centering
    \renewcommand{\arraystretch}{1.5}
    \begin{tabular}{|l|l|l|}
        \hline
        \textbf{Objective function \( f(x) \)}
        & \textbf{Minimum point \( x^\ast \)}
        & \textbf{Minimum value \( f(x^\ast) \)} \\
        \hline

        \( f(x) = \half x^2, \quad x \in \R \)
        & \( x^\ast = 0 \)
        & \( f(x^\ast) = 0 \) \\
        \hline

        \( f(x) = \half {(x - x_0)}^2, \quad x, x_0 \in \R \)
        & \( x^\ast = x_0 \)
        & \( f(x^\ast) = 0 \) \\
        \hline

        \( f(x) = \half {(x - x_0)}^2 + c, \quad x, x_0, c \in \R \)
        & \( x^\ast = x_0 \)
        & \( f(x^\ast) = c \) \\
        \hline

        \hypertarget{1D-quadratic}
        {\( f(x) = ax^2 + bx + c, \quad x, a, b, c \in \R, \; a > 0 \)}
        & \( x^\ast = -\cfrac{b}{2a} \)
        & \( f(x^\ast) = c - \cfrac{b^2}{4a} \)
        \rule[-13pt]{0pt}{36pt} \\
        \hline

        \hypertarget{line-point-minimisation}
        {\( f(t) = \cfrac{1}{2} {\big\Vert \u - t \v \big\Vert}^2, \quad t \in \R, \; \u, \v \in \R^d, \; \v \neq \zero \)}
        & \( t^\ast = \cfrac{\dotp{\u}{\v}}{\dotp{\v}{\v}} \)
        & \( f(t^\ast) = \cfrac{1}{2} \left( \norm{\u}^2 - \cfrac{{(\dotp{\u}{\v})}^2}{\dotp{\v}{\v}} \right) \)
        \rule[-20pt]{0pt}{48pt} \\
        \hline
    \end{tabular}
\end{table}

Note that \( \because f(t) \geq 0, \; \forall t \in \R \implies f(t^\ast) \geq 0 \), which gives the Cauchy-Schwarz inequality~\psecref{sec:cauchy-schwarz-inequality}.

\section{Optimisation methods}

\subsection{First-order methods}

Assume \( f \in \C^{1} \), and use \( \func{f}{\x + \p} = \func{f}{\x} + \dotp{\grad{f}{\x}}{\p} + \smalloh{\norm{\p}} \).

\subsection{Second-order methods}

Assume \( f \in \C^{2} \), and use \( \func{f}{\x + \p} = \func{f}{\x} + \dotp{\grad{f}{\x}}{\p} + \half \qf{\p}{\hess{f}{\x}} + \smalloh{\norm{\p}^2} \).

\section{Types of minimum}

\subsection{Global minimum}

\begin{definition}{Global minimum}{global-minimum}
    The point \( \xstar \in \R^d \) is a \textbf{global minimum} of the function \( f: D \subseteq \R^d \to \R \) if
    \begin{equation*}
        \func{f}{\xstar} \leq \func{f}{\x}, \quad \forall \x \in D
    \end{equation*}
\end{definition}

\subsection{Local minimum}

\begin{definition}{Local minimum}{}
    The point \( \xstar \in \R^d \) is a \textbf{local minimum} of the function \( f: \R^d \to \R \) if there exists a \( \delta > 0 \) such that for all \( \x \) in the \( \delta \)-neighborhood of \( \xstar \), we have \( \func{f}{\xstar} \leq \func{f}{\x} \), i.e.,
    \begin{equation*}
        \func{f}{\xstar} \leq \func{f}{\x}, \quad \forall \x \in B_{\delta}(\xstar)
    \end{equation*}
\end{definition}

\subsection{Strict local minimum}

\begin{definition}{Strict local minimum}{}
    The point \( \xstar \in \R^d \) is a \textbf{strict local minimum} of the function \( f: \R^d \to \R \) if there exists a \( \delta > 0 \) such that for all \( \x \) in the \( \delta \)-neighborhood of \( \xstar \) except \( \xstar \) itself, we have \( \func{f}{\xstar} < \func{f}{\x} \), i.e.,
    \begin{equation*}
        \func{f}{\xstar} < \func{f}{\x},
        \quad \forall \x \in B_{\delta}(\xstar) \setminus \set{\xstar}
    \end{equation*}
\end{definition}

\section{Necessary and sufficient conditions}

\subsection{First-order necessary condition for a local minimum}

\begin{theorem}{First-order necessary condition for a local minimum}{first-order-necessary-condition-for-a-local-minimum}
    For a function \( f \in \C^{1} \), if \( \xstar \) is a local minimum of \( f \), then \( \grad{f}{\xstar} = \zero \).
\end{theorem}

\subsection{Second-order necessary condition for a local minimum}

\begin{theorem}{Second-order necessary condition for a local minimum}{}
    For a function \( f \in \C^{2} \), if \( \xstar \) is a local minimum of \( f \), then \( \grad{f}{\xstar} = \zero, \; \hess{f}{\xstar} \succeq \zero \).
\end{theorem}

\subsection{Second-order sufficient condition for a strict local minimum}

\begin{theorem}{Second-order sufficient condition for a strict local minimum}{}
    For a function \( f \in \C^{2} \) such that \( \grad{f}{\xstar} = \zero, \; \hess{f}{\xstar} \succ \zero \), then \( \xstar \) is a strict local minimum of \( f \).
\end{theorem}

\subsection{First-order sufficient condition for a global minimum under convexity}

\begin{theorem}{First-order sufficient condition for a global minimum under convexity}{}
    For a convex function \( f \in \C^1 \) with \( \grad{f}{\xstar} = \zero \), then \( \xstar \) is a global minimum of \( f \).
\end{theorem}

\begin{proof}
    From the first-order condition for convexity~\pthmref{thm:first-order-condition-for-convexity}, we have
    \begin{align*}
        \func{f}{\y}
        & \geq
        \func{f}{\xstar} + \cancel{ \dotp{\grad{f}{\xstar}}{(\y - \xstar)} }
        , \quad \forall \y \in D
        \\
        \implies
        \func{f}{\y}
        & \geq
        \func{f}{\xstar}
        , \quad \forall \y \in D
    \end{align*}
    which is the definition of a global minimum~\pdefref{def:global-minimum}.
\end{proof}

\section{Convexity}

\begin{definition}{Convex set}{}
    A set \( S \subseteq \R^n \) is \textbf{convex} if
    \begin{equation*}
        \lambda \x + (1 - \lambda) \y \in S,
        \quad \forall \x, \y \in S,
        \; \lambda \in [0, 1]
    \end{equation*}
\end{definition}

\begin{definition}{Convex function}{}
    A function $f: D \subseteq \R^n \to \R$ is \textbf{convex} on a convex set \( D \) if
    \begin{equation*}
        \func{f}{\lambda \x + (1 - \lambda) \y}
        \leq
        \lambda \func{f}{\x} + (1 - \lambda) \func{f}{\y},
        \quad \forall \x, \y \in D,
        \; \lambda \in [0, 1]
    \end{equation*}
\end{definition}

\subsection{First-order condition for convexity}

\begin{theorem}{First-order condition for convexity}{first-order-condition-for-convexity}
    For a function \( f \in \C^1 \) with a convex domain \( D \), \( f \) is convex on \( D \) iff
    \begin{equation*}
        \func{f}{\y}
        \geq
        \func{f}{\x} + \dotp{\grad{f}{\x}}{(\y - \x)},
        \quad \forall \x, \y \in D
    \end{equation*}
\end{theorem}

\begin{proof}
    \( (\Rightarrow) \)
    Rearranging the definition of convexity, we get
    \begin{equation*}
        \frac{\func{f}{\lambda \x + (1 - \lambda) \y} - \func{f}{\x}}{\lambda}
        \leq
        \func{f}{\y} - \func{f}{\x},
        \quad \forall \x, \y \in D,
        \; \lambda \in \linterval{0}{1}
    \end{equation*}
    Taking the limit as \( \lambda \to 0^+ \), we have
    \begin{equation*}
        \dotp{\grad{f}{\x}}{(\y - \x)}
        =
        \lim_{\lambda \to 0^+} \frac{\func{f}{\lambda \x + (1 - \lambda) \y} - \func{f}{\x}}{\lambda}
        \leq
        \func{f}{\y} - \func{f}{\x},
        \quad \forall \x, \y \in D
    \end{equation*}
    which is the desired result.
\end{proof}

\begin{corollary}{Monotonicity of gradient \( \iff \) Convexity}{}
    For a function \( f \in \C^1 \) with a convex domain \( D \), \( f \) is convex on \( D \) iff
    \begin{equation*}
        \dotp{\big( \grad{f}{\x} - \grad{f}{\y} \big)}{\big( \x - \y \big)} \geq 0
        , \quad
        \forall \x, \y \in D
    \end{equation*}
\end{corollary}

\begin{proof}
    \( (\Rightarrow) \)
    From the first-order condition for convexity, we have
    \begin{align*}
        \func{f}{\y}
        & \geq
        \func{f}{\x} + {\grad{f}{\x}}^\top (\y - \x)
        , \quad
        \forall \x, \y \in D
        \\
        \func{f}{\x}
        & \geq
        \func{f}{\y} + {\grad{f}{\y}}^\top (\x - \y)
        , \quad
        \forall \x, \y \in D
    \end{align*}
    Adding the two inequalities, we get
    \begin{equation*}
        0 \geq {\big( \grad{f}{\x} - \grad{f}{\y} \big)}^\top (\y - \x)
        , \quad
        \forall \x, \y \in D
    \end{equation*}
    which is equivalent to the desired result.

    \( (\Leftarrow) \)
    Define \( \func{g}{t} = f(\x + t (\y - \x)), \;\; t \in [0, 1], \;\; \x, \y \in D \).

    Then, \( \func{g}{0} = \func{f}{\x}, \quad \func{g}{1} = \func{f}{\y}, \quad \func{g'}{t} = \dotp{(\grad{f}{\x + t (\y - \x)})}{(\y - \x)} \).

    From the given condition, we have
    \begin{align*}
        \func{g'}{t}
        & =
        {\big( \nabla f(\x + t (\y - \x)) - \grad{f}{\x} + \grad{f}{\x} \big)}^\top (\y - \x)
        \\ & =
        {\big( \nabla f(\x + t (\y - \x)) - \grad{f}{\x} \big)}^\top (\y - \x) + {\big( \grad{f}{\x} \big)}^\top (\y - \x)
        \geq
        {\big( \grad{f}{\x} \big)}^\top (\y - \x)
    \end{align*}
    for all \( t \in [0, 1] \).
    Integrating from \( 0 \) to \( 1 \), we get
    \begin{equation*}
        \func{f}{\y} - \func{f}{\x}
        =
        \func{g}{1} - \func{g}{0}
        =
        \int_{0}^{1} \func{g'}{t} \, dt
        \geq
        \int_{0}^{1} {\big( \grad{f}{\x} \big)}^\top (\y - \x) \, dt
        =
        {\big( \grad{f}{\x} \big)}^\top (\y - \x)
    \end{equation*}
    for all \( \x, \y \in D \), which is the first-order condition for convexity, and thereby \( f \) is convex.
\end{proof}

\subsection{Second-order condition for convexity}

\begin{theorem}{Second-order condition for convexity}{second-order-condition-for-convexity}
    For a function \( f \in \C^2 \) with a convex domain \( D \), \( f \) is convex on \( D \) iff
    \begin{equation*}
        \hess{f}{\x} \succeq \zero
        , \quad
        \forall \x \in D
    \end{equation*}
\end{theorem}

\begin{proof}
    \( (\Rightarrow) \)
    The first-order condition for convexity can be rewritten as
    \begin{equation*}
        \func{f}{\y} - \func{f}{\x} - {\grad{f}{\x}}^\top (\y - \x) \geq 0
        , \quad
        \forall \x, \y \in D
    \end{equation*}
    Using Taylor's theorem, we have
    \begin{equation*}
        \func{f}{\y} - \func{f}{\x} - {\grad{f}{\x}}^\top (\y - \x)
        =
        \half {(\y - \x)}^\top \hess{f}{\boldsymbol{\xi}} (\y - \x)
        , \quad
        \boldsymbol{\xi} = (1 - t) \x + t \y, \; t \in (0, 1)
    \end{equation*}
    for some \( t \in (0, 1) \).
    Therefore, we have
    \begin{equation*}
        {(\y - \x)}^\top \hess{f}{\boldsymbol{\xi}} (\y - \x) \geq 0
        , \quad
        \forall \x, \y \in D
    \end{equation*}
    which implies \( \hess{f}{\x} \succeq \zero, \; \forall \x \in D \).
    (Note: \( \boldsymbol{\xi} \in D \) since \( D \) is convex.)

    \( (\Leftarrow) \)
    From Taylor's theorem, we have
    \begin{equation*}
        \func{f}{\y} - \func{f}{\x} - {\grad{f}{\x}}^\top (\y - \x)
        =
        \half {(\y - \x)}^\top \hess{f}{\boldsymbol{\xi}} (\y - \x)
        , \quad
        \boldsymbol{\xi} = (1 - t) \x + t \y, \; t \in (0, 1)
    \end{equation*}
    for some \( t \in (0, 1) \).
    Since \( \hess{f}{\x} \succeq \zero, \; \forall \x \in D \), we have
    \begin{equation*}
        \func{f}{\y} - \func{f}{\x} - {\grad{f}{\x}}^\top (\y - \x) \geq 0
        , \quad
        \forall \x, \y \in D
    \end{equation*}
    which is the first-order condition for convexity, and thereby \( f \) is convex.
\end{proof}

\section{Strong convexity}

\begin{definition}{Strong convexity}{}
    A function \( f: D \subseteq \R^n \to \R \) on a convex domain \( D \) is \textbf{strongly convex} with parameter \( \mu > 0 \) if
    \begin{equation*}
        \func{f}{\lambda \x + (1 - \lambda) \y}
        \leq
        \lambda \func{f}{\x} + (1 - \lambda) \func{f}{\y}
        - \frac{\mu}{2} \lambda (1 - \lambda) \norm{\x - \y}^2,
        \quad \forall \x, \y \in \R^n, \; \lambda \in [0, 1]
    \end{equation*}
\end{definition}

\begin{corollary}{}{}
    A function \( f: D \subseteq \R^n \to \R \) on a convex domain \( D \) is strongly convex with parameter \( \mu > 0 \) iff the function \( \func{g}{\x} := \left( \func{f}{\x} - \frac{\mu}{2} \norm{\x}^2 \right) \) is convex on \( D \).
\end{corollary}

\begin{proof}
    Follows from the identity:
    \begin{align*}
        \lambda (1 - \lambda) \norm{\x - \y}^2
        & =
        \lambda (1 - \lambda) \Big( \norm{\y}^2 + \norm{\x}^2 - 2 \dotp{\x}{\y} \Big)
        \\ & =
        \lambda \norm{\y}^2 + \lambda \norm{\x}^2 - \lambda^2 \norm{\y}^2 - \norm{\lambda \x}^2 - 2 {(\lambda \x)}^\top \big( (1 - \lambda) \y \big)
        \\ & =
        \lambda \norm{\y}^2 + \lambda \norm{\x}^2 - \lambda^2 \norm{\y}^2 - \norm{\lambda \x + (1 - \lambda) \y}^2 + {(1 - \lambda)}^2 \norm{\y}^2
        \\ & =
        \cancel{ \lambda \norm{\y}^2 } + \lambda \norm{\x}^2 - \cancel{ \lambda^2 \norm{\y}^2 } - \norm{\lambda \x + (1 - \lambda) \y}^2 + \norm{\y}^2 - \cancel{ 2 } \lambda \norm{\y}^2 + \cancel{ \lambda^2 \norm{\y}^2 }
        \\ & =
        \lambda \norm{\x}^2 + (1 - \lambda) \norm{\y}^2 - \norm{\lambda \x + (1 - \lambda) \y}^2
    \end{align*}
\end{proof}

\subsection{First-order condition for strong convexity}

\begin{corollary}{First-order condition for strong convexity}{}
    For a function \( f \in \C^1 \) with a convex domain \( D \), \( f \) is strongly convex on \( D \) with parameter \( \mu > 0 \) iff
    \begin{equation*}
        \func{f}{\y}
        \geq
        \func{f}{\x} + {\grad{f}{\x}}^\top (\y - \x) + \frac{\mu}{2} \norm{\y - \x}^2,
        \quad \forall \x, \y \in D
    \end{equation*}
\end{corollary}

\begin{proof}
    Define \( \func{g}{\x} = \func{f}{\x} - \frac{\mu}{2} \norm{\x}^2 \).
    Then, \( \grad{g}{\x} = \grad{f}{\x} - \mu \x \).
    From the first-order condition for convexity~\pthmref{thm:first-order-condition-for-convexity}, we have that \( g \) is convex iff
    \begin{align*}
        \func{g}{\y}
        & \geq
        \func{g}{\x} + {\grad{g}{\x}}^\top (\y - \x),
        \quad \forall \x, \y \in D
        \\
        \implies
        \func{f}{\y} - \frac{\mu}{2} \norm{\y}^2
        & \geq
        \func{f}{\x} - \frac{\mu}{2} \norm{\x}^2 + {\big( \grad{f}{\x} - \mu \x \big)}^\top (\y - \x)
        \\
        \implies
        \func{f}{\y}
        & \geq
        \func{f}{\x} + {\grad{f}{\x}}^\top (\y - \x) + \frac{\mu}{2} \Big( \norm{\y}^2 - \norm{\x}^2 - 2 \dotp{\x}{(\y - \x)} \Big)
        \\ & =
        \func{f}{\x} + {\grad{f}{\x}}^\top (\y - \x) + \frac{\mu}{2} \Big( \norm{\y}^2 - \cancel{ \norm{\x}^2 } - 2 \dotp{\x}{\y} + \cancel{2} \norm{\x}^2 \Big)
        \\ & =
        \func{f}{\x} + {\grad{f}{\x}}^\top (\y - \x) + \frac{\mu}{2} \norm{\y - \x}^2
        , \quad
        \forall \x, \y \in D
    \end{align*}
\end{proof}

\subsection{Second-order condition for strong convexity}

\begin{corollary}{Second-order condition for strong convexity}{}
    For a function \( f \in \C^2 \) with a convex domain \( D \), \( f \) is strongly convex on \( D \) with parameter \( \mu > 0 \) iff
    \begin{equation*}
        \hess{f}{\x}
        \succeq
        \mu \I
        , \quad
        \forall \x \in D
    \end{equation*}
    i.e., the Hessian \( \hess{f}{\x} \) is positive definite with all eigenvalues at least \( \mu \), for all \( \x \in D \).
\end{corollary}

\begin{proof}
    Define \( \func{g}{\x} = \func{f}{\x} - \frac{\mu}{2} \norm{\x}^2 \).
    Then, \( \hess{g}{\x} = \hess{f}{\x} - \mu \I \).
    From the second-order condition for convexity~\pthmref{thm:second-order-condition-for-convexity}, we have that \( g \) is convex iff \( \hess{g}{\x} \succeq \zero, \; \forall \x \in D \), which is equivalent to the desired result.
\end{proof}

\chapter{Algorithmic design}

\section{Oracle}

An \textbf{oracle} is a procedure that provides information about the objective function \( f: \R^d \to \R \) at a given point \( \x \in \R^d \).
An \( n \)-th order oracle provides information up to the \( n \)-th derivative of \( f \) at \( \x \).
\begin{itemize}
    \item \textbf{Zero-order oracle:} Given \( \x \in \R^d \), returns the function value \( \func{f}{\x} \in \R \).

    \item \textbf{First-order oracle:} Given \( \x \in \R^d \), returns a tuple \( \big( \func{f}{\x}, \grad{f}{\x} \big) \in (\R, \R^d) \).

    \item \textbf{Second-order oracle:} Given \( \x \in \R^d \), returns a tuple \( \big( \func{f}{\x}, \grad{f}{\x}, \hess{f}{\x} \big) \in (\R, \R^d, \R^{d \times d}) \).
\end{itemize}

\section{Iterative algorithm template}

\begin{algorithm}[H]
    \caption{
        Iterative algorithm template for unconstrained minimisation
    }
    \SetAlgoLined{}
    \KwIn{
        First-order oracle for the objective function \( \func{f}{\x} \);
        Initial point \( \x^0 \in \R^d \);
    }
    \KwOut{
        Approximate solution to the unconstrained minimisation problem \( \displaystyle \argmin_{\x \in \R^d} \func{f}{\x} \)\;
    }

    \( k \leftarrow 0 \)\;

    \While{\( \x^k \) is not optimal}{
        Update the current point: \( \x^{k+1} = \func{\operatorname{ALGO}}{\x^k} \)\;

        \( k \leftarrow k + 1 \)\;
    }
    \Return{\( \x^k \)\;}
\end{algorithm}

\section{Line search methods}\label{sec:line-search}

\begin{algorithm}[H]
    \caption{
        Algorithm template for unconstrained minimisation using line search
    }
    \SetAlgoLined{}
    \KwIn{
        First-order oracle for the objective function \( \func{f}{\x} \);
        Initial point \( \x^0 \in \R^d \);
    }
    \KwOut{
        Approximate solution to the unconstrained minimisation problem \( \displaystyle \argmin_{\x \in \R^d} \func{f}{\x} \)\;
    }

    \( k \leftarrow 0 \)\;

    \While{\( \x^k \) is not optimal}{
        Choose a descent direction \( \p^k \) from the set of descent directions
        \begin{equation*}
            \func{\mathcal{DS}}{\x^k}
            =
            \set{\p \in \R^d \given \dotp{\grad{f}{\x^k}}{\p} < 0}
        \end{equation*}

        Choose a step length \( \alpha_k \geq 0 \)\;

        Update the current point: \( \x^{k+1} = \x^k + \alpha_k \p^k \)\;

        \( k \leftarrow k + 1 \)\;
    }
    \Return{\( \x^k \)\;}
\end{algorithm}

\paragraph{Gradient descent:}
Algorithms that use the gradient \( \grad{f}{\x^k} \) to compute the descent direction \( \p^k \).

\paragraph{Steepest descent method (Cauchy):}
Choose the descent direction as
\begin{equation*}
    \p^k = -\grad{f}{\x^k}
    ,\qquad \text{or normalized} \quad
    \p^k = - \frac{\grad{f}{\x^k}}{\norm{\grad{f}{\x^k}}}
\end{equation*}

\chapter{Exact line search}

Choose the step length \( \alpha_k \) in line search~\psecref{sec:line-search} by solving the one-dimensional optimisation problem
\begin{equation*}
    \alpha_k = \argmin_{\alpha \geq 0} g_k(\alpha),
    \qquad \text{where } \;
    g_k(\alpha) \triangleq f\left( \x^k + \alpha \p^k \right)
\end{equation*}

\section{Convex quadratic minimisation problem}

\paragraph{Problem:}
Consider the unconstrained quadratic minimisation problem
\begin{align*}
    \min_{\x \in \R^d} \func{f}{\x},
    \quad \text{where }
    \func{f}{\x} = \left( \half \qf{\x}{\Q} + \dotp{\h}{\x} + c \right),
    \quad \Q \succ \zero, \text{i.e.}, \Q \in \mathbf{S}_d^{++},
    \quad \h \in \R^d,
    \quad c \in \R
\end{align*}

\paragraph{Solution:}
Since \( \hess{f}{\x} = \Q \succ \zero \), \( f \) is strongly convex, and thereby has a unique global minimum which is also the unique local minimum.
The gradient of \( f \) can be computed as \( \grad{f}{\x} = \Q \x + \h \), and thereby from the first-order necessary condition for local minimum~\pthmref{thm:first-order-necessary-condition-for-a-local-minimum}, the local minimum \( \xstar \) must necessarily satisfy
\begin{equation*}
    \grad{f}{\xstar} = \Q \xstar + \h = \zero
    \quad \iff \quad
    \boxed{
        \xstar = -\Qinv \h
    }
\end{equation*}
where \( \Qinv \) exists since \( \Q \succ \zero \).

\paragraph{Challenge:}
Not allowed to compute \( \Qinv \) explicitly.
Matrix-vector products with \( \Q \) are allowed.

\paragraph{Algorithm design:}

To find the optimal step length \( \alpha_k \), consider
\begin{align*}
    g_k(\alpha)
    & \triangleq
    \func{f}{\x^k + \alpha \p^k}
    \\ & =
    \half \qf{\left( \x^k + \alpha \p^k \right)}{\Q}
    + \dotp{\h}{\!\left( \x^k + \alpha \p^k \right)}
    + c
    \\ & =
    \half \qf{\x^k}{\Q}
    + \alpha \qf{\x^k}{\Q}[\p^k]
    + \frac{\alpha^2}{2} \qf{\p^k}{\Q}
    + \dotp{\h}{\x^k}
    + \alpha \dotp{\h}{\p^k}
    + c
    \\ & =
    \alpha^2 \left( \half \qf{\p^k}{\Q} \right)
    + \alpha \left( \qf{\x^k}{\Q}[\p^k] + \dotp{\h}{\p^k} \right)
    + \left( \half \qf{\x^k}{\Q} + \dotp{\h}{\x^k} + c \right)
    \\ & =
    \alpha^2 \underbrace{ \left( \half \qf{\p^k}{\Q} \right) }_{p_k}
    + \alpha \underbrace{ \left( \dotp{\grad{f}{\x^k}}{\p^k} \right) }_{q_k}
    + \underbrace{ \Big( \func{f}{\x^k} \Big) }_{r_k}
\end{align*}
which is a quadratic function in \( \alpha \) with \( p_k > 0 \; \left( \because \Q \succ \zero \right) \) and \( q_k < 0 \; \left( \because \p^k \in \func{\mathcal{DS}}{\x^k} \right) \).

For the \hyperlink{1D-quadratic}{one-dimensional quadratic} case, we can then compute the minimum point and minimum value as
\begin{equation*}
    \alpha_k
    = -\frac{q_k}{2 p_k}
    \implies
    \boxed{
        \alpha_k
        = -\frac{\dotp{\grad{f}{\x^k}}{\p^k}}{\qf{\p^k}{\Q}}
    }
    > 0
\end{equation*}
\begin{equation*}
    \func{f}{\x^{k+1}}
    =
    g_k(\alpha_k)
    =
    r_k - \frac{q_k^2}{4 p_k}
    \implies
    \boxed{
        \func{f}{\x^{k+1}}
        =
        \func{f}{\x^k} - \frac{{\left( \dotp{\grad{f}{\x^k}}{\p^k} \right)}^2}{2 \qf{\p^k}{\Q}}
    }
\end{equation*}

For the choice \( \p^k = -\grad{f}{\x^k} \), we have
\begin{equation*}
    \alpha_k
    =
    \frac{\norm{\grad{f}{\x^k}}^2} {\qf{\grad{f}{\x^k}}{\Q}},
    \qquad
    \func{f}{\x^{k+1}}
    =
    \func{f}{\x^k} - \frac{\norm{\grad{f}{\x^k}}^4}{2 \qf{\grad{f}{\x^k}}{\Q}}
\end{equation*}

\paragraph{Analysis:}

For a local minimum \( \xstar \), and any \( \p \in \R^d \), we have
\begin{align*}
    \func{f}{\xstar + \p}
    & =
    \half \qf{(\xstar + \p)}{\Q} + \dotp{\h}{(\xstar + \p)} + c
    \\ & =
    \half \qf{\xstar}{\Q} + \qf{\xstar}{\Q}[\p] + \half \qf{\p}{\Q} + \dotp{\h}{\xstar} + \dotp{\h}{\p} + c
    \\ & =
    \left( \half \qf{\xstar}{\Q} + \dotp{\h}{\xstar} + c \right)
    + {\cancel{\left( \Q \xstar + \h \right)}}^\top \p
    + \half \qf{\p}{\Q}
    \\ & =
    \func{f}{\xstar} + \half \qf{\p}{\Q}
    \\
    \implies
    \func{f}{\x}
    & =
    \func{f}{\xstar} + \half \qf{(\x - \xstar)}{\Q},
    \quad \forall \x \in \R^d
\end{align*}

Now, consider the error at the \( k^{th} \) iteration defined as
\begin{align*}
    E_k
    & \triangleq
    \func{f}{\x^k} - \func{f}{\xstar}
    \\
    \implies
    E_k
    & =
    \half \qf{(\x^k - \xstar)}{\Q}
    \\
    \because
    \grad{f}{\xstar}
    & =
    \Q \xstar + \h
    =
    \zero
    \\
    \implies
    \grad{f}{\x^k}
    & =
    \Q \x^k + \h
    =
    \Q \left( \x^k - \xstar \right)
    \\
    \implies
    \left( \x^k - \xstar \right)
    & =
    \Qinv \, \grad{f}{\x^k}
    \\
    \implies
    E_k
    & =
    \half \dotp{\left( \x^k - \xstar \right)}{\grad{f}{\x^k}}
    =
    \half \dotp{\Big( \Qinv \, \grad{f}{\x^k} \Big)}{\grad{f}{\x^k}}
    \\ & =
    \half \qf{\grad{f}{\x^k}}{{\left( \Qinv \right)}^\top}
    =
    \half \qf{\grad{f}{\x^k}}{\Qinv}
\end{align*}

Now, the successive error reduction can be computed as
\begin{align*}
    E_k - E_{k+1}
    & =
    \func{f}{\x^k} - \func{f}{\x^{k+1}}
    =
    \frac{{\left( \dotp{\grad{f}{\x^k}}{\p^k} \right)}^2}{2 \qf{\p^k}{\Q}}
    \\
    \implies
    \frac{E_{k} - E_{k+1}}{E_k}
    & =
    \frac{{\left( \dotp{\grad{f}{\x^k}}{\p^k} \right)}^2}{ {\left( \grad{f}{\x^k} \right)}^\top \Qinv \, \grad{f}{\x^k} \; \qf{\p^k}{\Q}}
\end{align*}

For the choice \( \p^k = -\grad{f}{\x^k} \), we have
\begin{align*}
    \frac{E_{k} - E_{k+1}}{E_k}
    & =
    \frac{{\left( {\grad{f}{\x^k}}^\top \left( -\grad{f}{\x^k} \right) \right)}^2}{ {\left( \grad{f}{\x^k} \right)}^\top \Qinv \, \grad{f}{\x^k} \; {\left( -\grad{f}{\x^k} \right)}^\top \Q \left( -\grad{f}{\x^k} \right)}
    \\ & =
    \frac{{\Vert \grad{f}{\x^k} \Vert}^4}{ {\left( \grad{f}{\x^k} \right)}^\top \Qinv \, \grad{f}{\x^k} \; {\grad{f}{\x^k}}^\top \Q \grad{f}{\x^k}}
\end{align*}

From Kantorovich inequality~\psecref{sec:kantorovich-inequality}, we have
\begin{equation*}
    \frac{\norm{\z}^4}{(\qf{\z}{\Q})(\qf{\z}{\Qinv})}
    \geq
    \frac{4 \func{\lambda_{\min}}{\Q} \func{\lambda_{\max}}{\Q}}{{\left( \func{\lambda_{\min}}{\Q} + \func{\lambda_{\max}}{\Q} \right)}^2},
    \quad \forall \z \in \R^d \setminus \set{\zero}
\end{equation*}
where \( \func{\lambda_{\min}}{\Q} \) and \( \func{\lambda_{\max}}{\Q} \) are the minimum and maximum eigenvalues of \( \Q \), respectively.

Applying this with \( \z = \grad{f}{\x^k} \), we have
\begin{align*}
    \frac{E_k - E_{k+1}}{E_k}
    & \geq
    \frac{4 \func{\lambda_{\min}}{\Q} \func{\lambda_{\max}}{\Q}}{{\left( \func{\lambda_{\min}}{\Q} + \func{\lambda_{\max}}{\Q} \right)}^2}
    \\
    \implies
    \frac{E_{k+1}}{E_k}
    & \leq
    1 - \frac{4 \func{\lambda_{\min}}{\Q} \func{\lambda_{\max}}{\Q}}{{\left( \func{\lambda_{\min}}{\Q} + \func{\lambda_{\max}}{\Q} \right)}^2}
    =
    {\left( \frac{\func{\lambda_{\max}}{\Q} - \func{\lambda_{\min}}{\Q}}{\func{\lambda_{\max}}{\Q} + \func{\lambda_{\min}}{\Q}} \right)}^2
    \triangleq
    \rho
\end{align*}
\begin{equation*}
    \implies
    \frac{E_k}{E_0}
    =
    \prod_{i = 0}^{k-1} \frac{E_{i+1}}{E_i}
    \leq
    \prod_{i = 0}^{k-1} \rho
    =
    \rho^k
    \implies
    E_k
    \leq
    \rho^k E_0
\end{equation*}
where \( \rho \in \rinterval{0}{1} \; \because \Q \succ \zero, \; \func{\lambda_{\max}}{\Q} \geq \func{\lambda_{\min}}{\Q} > 0 \), and \( 4ab \leq {(a+b)}^2, \; \forall a, b \).

\chapter{Inexact line search}

\section{Sufficient decrease condition (Armijo condition)}

\begin{definition}{Sufficient decrease condition (Armijo condition)~\citep{Nocedal2006}}{}
    \begin{equation*}
        \func{f}{\x^k + \alpha_k \p^k}
        \leq
        \func{f}{\x^k} + c \alpha_k \dotp{\grad{f}{\x^k}}{\p^k},
        \quad 0 < c < 1
    \end{equation*}
\end{definition}

\subsection{Forward expansion}

Note that for sufficiently small \( \alpha_k \), the sufficient decrease condition is always satisfied since
\begin{align*}
    \func{f}{\x^k + \alpha_k \p^k}
    & =
    \func{f}{\x^k} + \alpha_k \dotp{\grad{f}{\x^k}}{\p^k} + \smalloh{\alpha_k}
    \\ & <
    \func{f}{\x^k} + c \alpha_k \dotp{\grad{f}{\x^k}}{\p^k},
    \qquad \because
    \dotp{\grad{f}{\x^k}}{\p^k} < 0,
    \quad 0 < c < 1
\end{align*}

\subsection{Backtracking line search}

Determine the largest number \( \alpha_k \) in the sequence \( \set{\beta^m \bar{\alpha}}_{m = 0}^{\infty} \) that satisfies the \textit{sufficient decrease condition}, where \( \bar{\alpha} > 0 \) is an initial step length and \( 0 < \beta < 1 \) is a scaling factor.

\section{Goldstein conditions / Armijo-Goldstein conditions}

\begin{definition}{Goldstein conditions / Armijo-Goldstein conditions~\citep{Nocedal2006}}{}
    \vspace*{-1em}
    \begin{align*}
        \func{f}{\x^k} + (1 - c) \alpha_k \dotp{\grad{f}{\x^k}}{\p^k}
        & \leq
        \overbrace{
            \func{f}{\x^k + \alpha_k \p^k}
            \leq
            \func{f}{\x^k} + c \alpha_k \dotp{\grad{f}{\x^k}}{\p^k}
        }^{\text{sufficient decrease}}
        \\
        \text{where } \;&
        0 < c < \half
    \end{align*}
\end{definition}

\subsection{Rate of convergence}

\begin{equation*}
    \sim
    R_k
    =
    \frac{2 \rho_1 (1 - \rho_2)}{L} \cos^2 \theta^{(k)}
\end{equation*}

\section{Wolfe conditions}

\begin{definition}{Wolfe conditions~\citep{Nocedal2006}}{}
    \vspace*{-1em}
    \begin{align*}
        \func{f}{\x^k + \alpha_k \p^k}
        & \leq
        \func{f}{\x^k} + c_1 \alpha_k \dotp{\grad{f}{\x^k}}{\p^k}
        \tag{sufficient decrease}
        \\
        \dotp{\grad{f}{\x^k + \alpha_k \p^k}}{\p^k}
        & \geq
        c_2 \dotp{\grad{f}{\x^k}}{\p^k}
        \tag{curvature}
        \\
        \text{where } \;&
        0 < c_1 < c_2 < 1
    \end{align*}
\end{definition}

\begin{equation*}
    R = \frac{c_1 (1 - c_2)}{L}
\end{equation*}

\subsection{Strong Wolfe conditions}

\begin{definition}{Strong Wolfe conditions~\citep{Nocedal2006}}{}
    \vspace*{-1em}
    \begin{align*}
        \func{f}{\x^k + \alpha_k \p^k}
        & \leq
        \func{f}{\x^k} + c_1 \alpha_k \dotp{\grad{f}{\x^k}}{\p^k}
        \tag{sufficient decrease}
        \\
        \abs{\dotp{\grad{f}{\x^k + \alpha_k \p^k}}{\p^k}}
        & \leq
        c_2 \abs{\dotp{\grad{f}{\x^k}}{\p^k}}
        \tag{strong curvature}
        \\
        \text{where } \;&
        0 < c_1 < c_2 < 1
    \end{align*}
\end{definition}

\section{Sufficient decrease lemma}

\begin{lemma}{Sufficient decrease lemma}{}
    Suppose \( f \in \C^{1}_{L} \) is bounded below, then \( \forall \x \in \R^n, \; \alpha \in \linterval[scaled]{0}{\frac{1}{L}} \), the gradient descent update \( \xhat = \x - \alpha \grad{f}{\x} \) yields a sufficient decrease in the function value:
    \begin{equation*}
        \func{f}{\xhat}
        \leq
        \func{f}{\x} - \frac{\alpha}{2} \norm{\grad{f}{\x}}^2
    \end{equation*}
\end{lemma}

\begin{equation*}
    \frac{\func{f}{\x^k} - \func{f}{\x^{k+1}}}{\norm{\grad{f}{\x^k}}^2}
    \geq R_k \geq R > 0
\end{equation*}

\section{Global convergence criterion}

Starting from an arbitrary \( \x^0 \in \R^d \), if a scheme generates a sequence \( \set{\x^k}_{k = 1}^{\infty} \) such that
\begin{equation*}
    \lim_{k \to \infty} \norm{\grad{f}{\x^k}} = 0
\end{equation*}
then the scheme is said to exhibit \textbf{global convergence}.

The sequence asymptotically approaches a \textbf{stationary point} \( \xstar \), i.e., \( \grad{f}{\xstar} = \zero \).
Without further assumptions on \( f \), nothing can be said about \( \xstar \) being a local or global minimum, maximum, or a saddle point.
The `global' refers to the fact that the initial point \( \x^0 \) is arbitrary, and not about the limit point.

\chapter{Second-order methods}

\begin{equation*}
    \x^{k+1}
    =
    \x^k - \hessinv{f}{\x} \grad{f}{\x^k},
    \qquad
    \hess{f}{\x^k} \succ \zero
\end{equation*}

\paragraph{Newton direction:}
Refers to the descent direction
\begin{equation*}
    \p^k = -\hessinv{f}{\x^k} \grad{f}{\x^k},
    \qquad \hess{f}{\x^k} \succ \zero
\end{equation*}

This satisfies the condition for descent direction since
\begin{align*}
    \hess{f}{\x^k}
    &
    \succ \zero
    \implies
    \hessinv{f}{\x^k}
    \succ \zero
    \\
    \implies
    \dotp{\grad{f}{\x^k}}{\p^k}
    & =
    - {\left( \grad{f}{\x^k} \right)}^\top \hessinv{f}{\x^k} \grad{f}{\x^k}
    < 0
\end{align*}

From the fundamental theorem of calculus~\psecref{sec:fundamental-theorem-of-calculus}, with \( \func{g}{t} \triangleq \grad{f}{\x + t(\y - \x)}, \; t \in [0, 1] \), we have
\begin{equation*}
    \func{g}{1} - \func{g}{0}
    =
    \int_{0}^{1} \func{g'}{t} \, dt
    \implies
    \boxed{
        \grad{f}{\y} - \grad{f}{\x}
        =
        \int_{0}^{1} \hess{f}{\x + t(\y - \x)} (\y - \x) \, dt,
        \quad \forall \x, \y \in \R^d
    }
\end{equation*}

For a stationary point \( \xstar \), i.e., \( \grad{f}{\xstar} = \zero \), we have
\begin{equation*}
    \implies
    \grad{f}{\x}
    =
    \int_{0}^{1} \hess{f}{\xstar + t(\x - \xstar)} (\x - \xstar) \, dt,
    \quad \forall \x \in \R^d
\end{equation*}

Suppose \( \hess{f}{\xstar} \succ \zero \), ensuring that \( \xstar \) is a local minimum.
\begin{equation*}
    \implies
    \exists l > 0 \text{ such that } \hess{f}{\xstar} \succeq l \, \I
\end{equation*}

Assuming \( f \in \C^2_M \), we have
\begin{equation*}
    \boxed{
        \norm{\hess{f}{\y} - \hess{f}{\x}}
        \leq
        M \norm{\y - \x},
        \quad \forall \x, \y \in \R^d
    }
\end{equation*}

From Taylor's theorem, we have
\begin{equation*}
    \func{f}{\y}
    =
    \func{f}{\x}
    + \dotp{\grad{f}{\x}}{(\y - \x)}
    + \half \qf{(\y - \x)}{\hess{f}{\x}}
    + \smalloh{\norm{\y - \x}^2},
    \quad \forall \x, \y \in \R^d
\end{equation*}

Taking the derivative with respect to \( \y \) on both sides, we have
\begin{equation*}
    \implies
    \grad{f}{\y}
    =
    \grad{f}{\x}
    + \hess{f}{\x} (\y - \x)
    + \smalloh{\norm{\y - \x}},
    \quad \forall \x, \y \in \R^d
\end{equation*}

\begin{align*}
    \implies
    \norm{
        \grad{f}{\y}
        - \grad{f}{\x}
        - \hess{f}{\x} (\y - \x)
    }
    & =
    \norm{
        \int_{0}^{1} \Big( \hess{f}{\x + t(\y - \x)} - \hess{f}{\x} \Big) (\y - \x) \, dt
    }
    \\ & \leq
    \int_{0}^{1} \norm{
        \Big( \hess{f}{\x + t(\y - \x)} - \hess{f}{\x} \Big) (\y - \x)
    } \, dt
    \\ & \leq
    \int_{0}^{1}
    \norm{\hess{f}{\x + t(\y - \x)} - \hess{f}{\x}}
    \norm{\y - \x}
    \, dt
    \\ & \leq
    \int_{0}^{1}
    M \norm{\cancel{\x} + t(\y - \x) - \cancel{\x}}
    \norm{\y - \x}
    \, dt
    \\ & =
    \int_{0}^{1}
    M
    \norm{\y - \x}^2
    \, t
    \, dt
    =
    M
    \norm{\y - \x}^2
    \int_{0}^{1}
    t
    \, dt
    =
    \frac{M}{2}
    \norm{\y - \x}^2
\end{align*}
\begin{equation*}
    \implies
    \boxed{
        \norm{
            \grad{f}{\y}
            - \grad{f}{\x}
            - \hess{f}{\x} (\y - \x)
        }
        \leq
        \frac{M}{2} \norm{\y - \x}^2,
        \quad \forall \x, \y \in \R^d
    }
\end{equation*}

Now, since \( - \norm{\A} \, \I \preceq \A \preceq \norm{\A} \, \I, \quad \forall \A \in \SD \), we have
\begin{align*}
    - \norm{\hess{f}{\y} - \hess{f}{\x}} \, \I
    \preceq
    \hess{f}{\y} - \hess{f}{\x}
    \preceq
    \norm{\hess{f}{\y} - \hess{f}{\x}} \, \I,
    \quad \forall \x, \y \in \R^d
    \\
    \implies
    - M \norm{\y - \x} \, \I
    \preceq
    \hess{f}{\y} - \hess{f}{\x}
    \preceq
    M \norm{\y - \x} \, \I,
    \quad \forall \x, \y \in \R^d
    \\
    \implies
    \boxed{
        \hess{f}{\x} - M \norm{\y - \x} \, \I
        \preceq
        \hess{f}{\y}
        \preceq
        \hess{f}{\x} + M \norm{\y - \x} \, \I,
        \quad \forall \x, \y \in \R^d
    }
\end{align*}

Let \( r_k \triangleq \norm{\x^k - \xstar} \).
Then, we have
\begin{align*}
    \implies
    &
    \hess{f}{\xstar} - M r_k \, \I
    \preceq
    \hess{f}{\x^k}
    \preceq
    \hess{f}{\xstar} + M r_k \, \I
    \\
    \implies
    &
    (l - M r_k) \, \I
    \preceq
    \hess{f}{\x^k}
    \\
    \therefore
    \quad
    &
    \text{If }
    (l - M r_k) > 0
    \implies
    \hess{f}{\x^k}
    \succ
    \zero
    \\
    \text{i.e.,}
    \quad
    &
    \text{If }
    \boxed{
        r_k < \frac{l}{M}
    }
    \implies
    \hess{f}{\x^k}
    \succ
    \zero
    \\
    \text{Also, }
    \quad
    &
    (l - M r_k) \, \I
    \preceq
    \hess{f}{\x^k}
    \implies
    \frac{1}{(l - M r_k)} \, \I
    \succeq
    \hessinv{f}{\x^k}
    \\
    \text{i.e.,}
    \quad
    &
    \hessinv{f}{\x^k}
    \preceq
    \frac{1}{(l - M r_k)} \, \I
    \implies
    \boxed{
        \norm{\hessinv{f}{\x^k}}
        \leq
        \frac{1}{(l - M r_k)}
    }
\end{align*}

Now,
\begin{align*}
    r_{k+1}
    & =
    \norm{\x^{k+1} - \xstar}
    =
    \norm{\left( \x^k - \hessinv{f}{\x^k} \grad{f}{\x^k} \right) - \xstar}
    \\ & =
    \norm{\x^k - \xstar - \hessinv{f}{\x^k} \grad{f}{\x^k}}
    \\ & =
    \norm{\hessinv{f}{\x^k} \Big( \hess{f}{\x^k} (\x^k - \xstar) - \grad{f}{\x^k} \Big)}
    \\ & \leq
    \norm{\hessinv{f}{\x^k}} \,
    \norm{\hess{f}{\x^k} (\x^k - \xstar) - \grad{f}{\x^k}}
\end{align*}
\begin{align*}
    \implies
    &
    \norm{\hess{f}{\x^k} (\x^k - \xstar) - \grad{f}{\x^k}}
    \\ & =
    \norm{\hess{f}{\x^k} (\x^k - \xstar) - \int_{0}^{1} \hess{f}{\xstar + t(\x^k - \xstar)} (\x^k - \xstar) \, dt}
    \\ & =
    \norm{\int_{0}^{1} \hess{f}{\x^k} (\x^k - \xstar)  \, dt - \int_{0}^{1} \hess{f}{\xstar + t(\x^k - \xstar)} (\x^k - \xstar) \, dt}
    \\ & =
    \norm{\int_{0}^{1} \Big( \hess{f}{\x^k} - \hess{f}{\xstar + t(\x^k - \xstar)} \Big) (\x^k - \xstar) \, dt}
    \\ & \leq
    \int_{0}^{1} \norm{\Big( \hess{f}{\x^k} - \hess{f}{\xstar + t(\x^k - \xstar)} \Big) (\x^k - \xstar)} \, dt
    \\ & \leq
    \int_{0}^{1} \norm[\Big]{\hess{f}{\x^k} - \hess{f}{\xstar + t(\x^k - \xstar)}} \, \norm{\x^k - \xstar} \, dt
    \\ & \leq
    \int_{0}^{1} M \norm[\Big]{\x^k - \xstar - t(\x^k - \xstar)} \, \norm{\x^k - \xstar} \, dt
    \\ & =
    \int_{0}^{1} M (1 - t) \norm{\x^k - \xstar}^2 \, dt
    \\ & =
    M
    \norm{\x^k - \xstar}^2
    \int_{0}^{1} (1 - t) \, dt
    =
    \frac{M}{2}
    r_k^2
\end{align*}
\begin{equation*}
    \implies
    \boxed{
        r_{k+1}
        \leq
        \frac{M}{2}
        \norm{\hessinv{f}{\x^k}}
        r_k^2
    }
    \implies
    r_{k+1}
    \leq
    \frac{M}{2 (l - M r_k)}
    r_k^2
\end{equation*}

Thereby, rate of convergence for Newton's method is quadratic.

Now, we want \( r_{k+1} < r_k \) to ensure successive iterates converge to \( \xstar \), thereby we get
\begin{equation*}
    \frac{M}{2 (l - M r_k)}
    r_k^2
    <
    r_k
    \implies
    M r_k
    <
    2 l - 2 M r_k
    \implies
    3 M r_k
    <
    2 l
    \implies
    \boxed{
        r_k
        <
        \frac{2 l}{3 M}
    }
\end{equation*}

Note that \( \displaystyle r_k < \frac{2 l}{3 M} \implies r_k < \frac{l}{M} \), ensuring \( \hess{f}{\x^k} \succ \zero \).

Thereby, if we start with \( \displaystyle \boxed{ r_0 < \frac{2 l}{3 M} } \), then since \( \displaystyle r_k < r_{k - 1} < \cdots < r_1 < r_0 < \frac{2 l}{3 M} \), this ensures that \( \displaystyle r_k < \frac{2 l}{3 M} \).
Thereby, we have \( \hess{f}{\x^k} \succ \zero \) throughout the iterations, ensuring that \( \p^k \) is a descent direction, thereby the algorithm converges to the local minimum \( \xstar \).

\chapter{Conjugate gradient methods}

\section{\texorpdfstring{\( \Q \)--conjugacy}{Q-conjugacy}}

\begin{definition}{\( \Q \)--conjugacy / \( \Q \)--orthogonality}{}
    Given some real symmetric positive definite matrix \( \Q \in \PD \), a pair of distinct vectors \( \u, \v \in \R^d, \; \u \neq \v \) are said to be \textbf{conjugates with respect to \( \Q \)} \text{ OR } \textbf{\( \Q \)--conjugates} if
    \vspace{-0.5em}
    \begin{equation*}
        \qf{\u}{\Q}[\v] = 0
    \end{equation*}
\end{definition}

\subsection{Properties}

\begin{itemize}
    \item\label{item:orthogonality-implies-q-conjugacy}
        \textit{Generalisation of orthogonality}:\\
        If \( \Q = \I \), then \( \Q \)--conjugacy reduces to orthogonality, i.e.,
        \(
            \Q = \I
            \implies
            \qf{\u}{\Q}[\v] = \dotp{\u}{\v} = 0
        \).

    \item \textit{Symmetry}:
        \(
            \qf{\u}{\Q}[\v] = 0
            \iff
            \qf{\v}{\Q}[\u] = 0,
            \quad \because \Q = \Q^\top
        \)

    \item Not transitive.
        \(
            \qf{\x}{\Q}[\y] = 0,
            \quad
            \qf{\y}{\Q}[\z] = 0
            \notimplies
            \qf{\x}{\Q}[\z] = 0
        \)

    \item The zero vector \( \zero \) is \( \Q \)--conjugate to every vector in \( \R^d \).
        \(
            \qf{\zero}{\Q}[\v]
            =
            \qf{\v}{\Q}[\zero]
            = 0,
            \quad \forall \v \in \R^d
        \)
\end{itemize}

\begin{definition}{Mutually \( \Q \)--conjugate vectors}{}
    A set of distinct vectors \( \set{ \u_i }_{i = 1}^n, \; \u_i \in \R^d, \; \u_i \neq \u_j, \; \forall i \in \set{1, 2, \ldots, n} \) are said to be \textbf{mutually conjugate with respect to \( \Q \)} \text{ OR } \textbf{pairwise/mutually \( \Q \)--conjugate} if
    \vspace{-0.5em}
    \begin{equation*}
        \qf{\u_i}{\Q}[\u_j] = 0,
        \quad \forall i, j \in \set{1, 2, \ldots, n}, \; i \neq j
    \end{equation*}
\end{definition}

\begin{theorem}{Linear independence of non-zero mutually \( \Q \)--conjugate vectors}{}
    Any set of non-zero mutually \( \Q \)--conjugate vectors are linearly independent.
\end{theorem}

\begin{proof}
    Let \( \set{ \u_i }_{i = 1}^n \) be a set of non-zero mutually \( \Q \)--conjugate vectors.
    Define \( \z \triangleq \sum_{i = 1}^{n} \beta_i \u_i, \; \beta_i \in \R \).

    Then, for any \( l \in \set{1, 2, \ldots, n} \), we have
    \begin{equation*}
        \qf{\u_l}{\Q}[\z]
        =
        \qf{\u_l}{\Q}[\left( \sum_{i = 1}^{n} \beta_i \u_i \right)]
        =
        \sum_{i = 1}^{n} \beta_i \qf{\u_l}{\Q}[\u_i]
        =
        \beta_l \qf{\u_l}{\Q}
        + \sum_{i \neq l} \beta_i \cancel{ \qf{\u_l}{\Q}[\u_i] }
        =
        \beta_l \qf{\u_l}{\Q}
    \end{equation*}

    From this, we can see that if \( \z = \zero \), then \( \qf{\u_l}{\Q}[\z] = \zero = \beta_l \qf{\u_l}{\Q} \).
    Since \( \qf{\u_l}{\Q} > 0, \because \Q \succ \zero, \u_l \neq \zero \), we have that \( \beta_i = 0, \; \forall i \in \set{1, 2, \ldots, n} \).
    Thereby, the vectors \( \set{ \u_i }_{i = 1}^n \) are linearly independent.
\end{proof}

\begin{proof}
    Proof by contradiction.

    Suppose \( \set{ \u_i }_{i = 1}^n \) are linearly dependent.
    Then, there exists some \( l \in \set{1, 2, \ldots, n} \) such that
    \begin{equation*}
        \u_l
        =
        \sum_{i \neq l} \beta_i \u_i,
        \quad \beta_i \in \R
    \end{equation*}

    Then, we have
    \begin{equation*}
        \qf{\u_l}{\Q}
        =
        \qf{\left( \sum_{i \neq l} \beta_i \u_i \right)}{\Q}[\u_l]
        =
        \sum_{i \neq l} \beta_i \qf{\u_i}{\Q}[\u_l]
        =
        \sum_{i \neq l} \beta_i \cdot 0
        =
        0
    \end{equation*}

    However, since \( \Q \succ \zero, \u_l \neq \zero \), we have \( \qf{\u_l}{\Q} > 0 \), which is a contradiction.

    Thereby, the vectors \( \set{ \u_i }_{i = 1}^n \) are linearly independent.
\end{proof}

\begin{corollary}{}{}
    Any set of non-zero mutually orthogonal vectors are linearly independent.
\end{corollary}

\section{Conjugate direction methods}

\begin{algorithm}[H]
    \caption{
        Conjugate direction algorithm for unconstrained minimisation
        of a differentiable function \( f : \R^d \to \R \), \( f \in \C^1 \).
    }
    \SetAlgoLined{}
    \KwIn{
        First-order oracle for the objective function \( \func{f}{\x} \);
        Initial point \( \x^0 \in \R^d \);

        A set of \( d \) mutually \( \Q \)--conjugate descent directions \( \set{\u_i}_{i = 0}^{d - 1}, \; \u_i \in \R^d, \; \forall i \in \set{0, 1, \ldots, d - 1} \)\;
        \vspace{-0.5em}
        \begin{equation*}
            \qf{\u_i}{\Q}[\u_j] = 0,
            \quad \forall i, j \in \set{0, 1, \ldots, d - 1}, \; i \neq j
        \end{equation*}
    }
    \KwOut{
        Approximate solution to the unconstrained minimisation problem \( \displaystyle \argmin_{\x \in \R^d} \func{f}{\x} \)\;
    }

    \( k \leftarrow 0 \)\;

    \While{\( \x^k \) is not optimal}{
        Choose step length \( \alpha_k \) by line search along direction \( \u_k \)\;

        Update the current point: \( \x^{k+1} = \x^k + \alpha_k \u_k \)\;

        \( k \leftarrow k + 1 \)\;
    }
    \Return{\( \x^k \)\;}
\end{algorithm}

\begin{algorithm}[H]
    \caption{
        Conjugate direction algorithm for unconstrained minimisation of a convex quadratic function \( f: \R^d \to \R \).
        \vspace{-1em}
        \begin{equation*}
            \func{f}{\x}
            =
            \half \qf{\x}{\Q} + \dotp{\h}{\x},
            \quad \Q \in \PD, \; \h \in \R^d
        \end{equation*}
    }
    \SetAlgoLined{}
    \KwIn{
        First-order oracle for the objective function \( \func{f}{\x} \);
        Initial point \( \x^0 \in \R^d \);

        A set of \( d \) mutually \( \Q \)--conjugate descent directions \( \set{\u_i}_{i = 0}^{d - 1}, \; \u_i \in \R^d, \; \forall i \in \set{0, 1, \ldots, d - 1} \)\;
        \vspace{-0.5em}
        \begin{equation*}
            \qf{\u_i}{\Q}[\u_j] = 0,
            \quad \forall i, j \in \set{0, 1, \ldots, d - 1}, \; i \neq j
        \end{equation*}
    }
    \KwOut{
        Exact solution to the unconstrained minimisation problem \( \displaystyle \argmin_{\x \in \R^d} \func{f}{\x} \)\;
    }

    \( k \leftarrow 0 \)\;

    \While{\( k < d \) \emph{ AND } \( \x^k \) is not optimal}{
        Choose step length \( \alpha_k \) by exact line search along direction \( \u_k \) as
        \vspace{-0.5em}
        \begin{equation*}
            \alpha_k
            =
            - \frac{\dotp{\grad{f}{\x^k}}{\u_k}}{\qf{\u_k}{\Q}}
        \end{equation*}

        \vspace{-0.5em}
        Update the current point: \( \x^{k+1} = \x^k + \alpha_k \u_k \)\;

        \( k \leftarrow k + 1 \)\;
    }
    \Return{\( \x^k \)\;}
\end{algorithm}

\paragraph{Finite termination property:}
For a convex quadratic function \( f: \R^d \to \R \), the conjugate direction method terminates in at most \( d \) iterations.

\begin{theorem}{Expanding subspace theorem~\citep{Nocedal2006}}{expanding-subspace-theorem}
    For a convex quadratic function \( f: \R^d \to \R \), the sequence of iterates \( \set{\x^k}_{k = 0}^{d} \) generated by the conjugate direction method satisfies
    \vspace{-0.5em}
    \begin{equation*}
        \begin{aligned}
            \x^k
            & =
            \argmin_{\x \in S_k} \func{f}{\x}
            \\
            \quad \text{where }
            S_k
            & \triangleq
            \set{\x \given \x = \x^0 + \p, \; \p \in \spanset{\u_i}_{i = 0}^{k - 1}}
        \end{aligned},
        \quad \forall k \in \set{1, 2, \ldots, d}
    \end{equation*}

    \vspace{-0.5em}
    i.e., the current iterate \( \x^k \) is the minimiser of \( f \) over the affine subspace spanned by the initial point \( \x^0 \) and the first \( k \) conjugate directions \( \set{\u_i}_{i = 0}^{k - 1} \), for all \( k \in \set{1, 2, \ldots, d} \).
\end{theorem}

\begin{proof}
    Define
    \begin{equation*}
        \begin{aligned}
            \U^{(k)}
            & \triangleq
            \begin{bmatrix}
                \u_0 & \u_1 & \cdots & \u_{k - 1}
            \end{bmatrix}
            \in \R^{d \times k}
            \\
            \gamma^{(k)}
            & \triangleq
            \begin{bmatrix}
                \gamma_0 & \gamma_1 & \cdots & \gamma_{k - 1}
            \end{bmatrix}^\top
            \in \R^k
        \end{aligned}
        , \quad \forall k \in \set{1, 2, \ldots, d}
    \end{equation*}
    and
    \begin{equation*}
        \func{g}{\gamma^{(k)}}
        \triangleq
        \func{f}{\x^0 + \U^{(k)} \gamma^{(k)}},
        \quad \forall k \in \set{1, 2, \ldots, d}
    \end{equation*}
    Then, since \( \set{\u_i}_{i = 0}^{d - 1} \) form a basis for \( \R^d \), the matrix \( \U^{(k)} \) has full column rank, i.e., \( \rank{\U^{(k)}} = k, \; \forall k \in \set{1, 2, \ldots, d} \).
    Thereby, for any \( \x \in S_k \), there exists a unique \( \gamma^{(k)} \in \R^k \) such that \( \x = \x^0 + \U^{(k)} \gamma^{(k)} \), for all \( k \in \set{1, 2, \ldots, d} \).
    Hence, we can see that the constrained minimisation problem with \( f \) over the affine subspace \( S_k \) is equivalent to the unconstrained minimisation problem with \( g \) over \( \R^k \), i.e.,
    \vspace{-0.5em}
    \begin{equation*}
        \min_{\x \in S_k} \func{f}{\x}
        \equiv
        \min_{\gamma^{(k)} \in \R^k} \func{g}{\gamma^{(k)}},
        \quad \forall k \in \set{1, 2, \ldots, d}
    \end{equation*}

    Applying Taylor's theorem~\psecref{sec:multivariate-taylor-theorem} to \( g \) at \( \gamma^{(k)} = \zero \), we have
    \begin{align*}
        \func{g}{\gamma^{(k)}}
        & =
        \func{g}{\zero}
        + \dotp{\grad{g}{\zero}}{\gamma^{(k)}}
        + \half \qf{\gamma^{(k)}}{\hess{g}{\zero}},
        \quad \forall k \in \set{1, 2, \ldots, d}
        \\
        \implies
        \grad{g}{\gamma^{(k)}}
        & =
        \dotp{\U^{(k)}}{\grad{f}{\x^0}}
        + \qf{\U^{(k)}}{\Q} \gamma^{(k)},
        \quad \forall k \in \set{1, 2, \ldots, d}
        \\
        \implies
        \hess{g}{\gamma^{(k)}}
        & =
        \qf{\U^{(k)}}{\Q},
        \quad \forall k \in \set{1, 2, \ldots, d}
    \end{align*}

    Since \( \set{\u_i}_{i = 0}^{d - 1} \) are mutually \( \Q \)--conjugate, it follows that \( \U^{(k)} \) is an orthogonal matrix, i.e., \( \dotp{\U^{(k)}}{\U^{(k)}} = \outp{\U^{(k)}}{\U^{(k)}} = \I, \; \forall k \in \set{1, 2, \ldots, d} \).
    Thereby, we have that \( \qf{\U^{(k)}}{\Q} \) is a diagonal matrix with positive diagonal entries, i.e., \( \hess{g}{\gamma^{(k)}} = \diag{\qf{\u_i}{\Q}}_{i = 0}^{k - 1} \succ \zero, \; \because \Q \succ \zero, \; \forall k \in \set{1, 2, \ldots, d} \).
    \begin{align*}
        \implies
        {\gamma^{(k)}}^\ast
        & =
        \argmin_{\gamma^{(k)} \in \R^k} \func{g}{\gamma^{(k)}}
        =
        - \hessinv{g}{\zero} \grad{g}{\zero}
        \\ & =
        - \inv{\left[ \qf{\U^{(k)}}{\Q} \right]} \dotp{\U^{(k)}}{\grad{f}{\x^0}}
        \\
        \implies
        {\gamma^{(k)}}^\ast_i
        & =
        - \frac{\dotp{\u_i}{\grad{f}{\x^0}}}{\qf{\u_i}{\Q}},
        \quad \forall i \in \set{0, 1, \ldots, k - 1}, \; \forall k \in \set{1, 2, \ldots, d}
    \end{align*}
    \begin{align*}
        \because
        \grad{g}{{\gamma^{(k)}}^\ast}
        & =
        \dotp{\U^{(k)}}{\grad{f}{\x^0 + Q \U^{(k)} \gamma}}
        =
        \zero
        \\
        \implies
        \dotp{\U^{(k)}}{\grad{f}{\x^k}}
        & =
        \zero,
        \quad \forall k \in \set{1, 2, \ldots, d}
    \end{align*}
    \begin{equation}\label{eq:conjugate-direction-orthogonality}
        \therefore
        \boxed{
            \dotp{\u_i}{\grad{f}{\x^k}}
            = 0,
            \quad \forall i \in \set{0, 1, \ldots, k - 1}, \; \forall k \in \set{1, 2, \ldots, d}
        }
    \end{equation}
    Thereby, we have that the gradient \( \grad{f}{\x^k} \) is orthogonal to the subspace spanned by the first \( k \) conjugate directions \( \set{\u_i}_{i = 0}^{k - 1} \), for all \( k \in \set{1, 2, \ldots, d} \).
\end{proof}

\section{Conjugate gradient methods}

\begin{algorithm}[H]\label{alg:preliminary-conjugate-gradient-method}
    \caption{
        {}~\citep{Nocedal2006}
        (Preliminary) Conjugate gradient algorithm for unconstrained minimisation of a convex quadratic function \( f: \R^d \to \R \).
        \vspace{-1em}
        \begin{equation*}
            \func{f}{\x}
            =
            \half \qf{\x}{\Q} + \dotp{\h}{\x},
            \quad \Q \in \PD, \; \h \in \R^d
        \end{equation*}
    }
    \SetAlgoLined{}
    \KwIn{
        First-order oracle for the objective function \( \func{f}{\x} \);
        Initial point \( \x^0 \in \R^d \);
    }
    \KwOut{
        Exact solution to the unconstrained minimisation problem \( \displaystyle \argmin_{\x \in \R^d} \func{f}{\x} \)\;
    }

    \( k \leftarrow 0 \)\;

    \( \u_0 \leftarrow -\grad{f}{\x^0} \)\;

    \While{\( k < d \) \emph{ AND } \( \x^k \) is not optimal}{
        Choose step length \( \alpha_k \) by exact line search along direction \( \u_k \) as
        \vspace{-0.5em}
        \begin{equation*}
            \alpha_k
            =
            - \frac{\dotp{\grad{f}{\x^k}}{\u_k}}{\qf{\u_k}{\Q}}
        \end{equation*}

        \vspace{-0.5em}
        Update the current point: \( \x^{k+1} = \x^k + \alpha_k \u_k \)\;

        Find the new gradient by using either the first-order oracle or the recurrence relation
        \vspace{-0.5em}
        \begin{equation*}
            \grad{f}{\x^{k+1}}
            =
            \grad{f}{\x^k}
            + \alpha_k \Q \u_k
        \end{equation*}
        or the quadratic structure of \( f \)
        \vspace{-0.5em}
        \begin{equation*}
            \grad{f}{\x^{k+1}}
            =
            \Q \x^{k+1} + \h
        \end{equation*}

        Find the next conjugate direction as
        \vspace{-0.5em}
        \begin{equation*}
            \u_{k+1}
            =
            -\grad{f}{\x^{k+1}}
            + \beta_k \u_k
        \end{equation*}
        such that \( \u_{k+1} \) is \( \Q \)--conjugate to \( \u_k \), thereby
        \vspace{-0.5em}
        \begin{equation*}
            \beta_k
            =
            \frac{\qf{\grad{f}{\x^{k+1}}}{\Q}[\u_k]}{\qf{\u_k}{\Q}}
        \end{equation*}

        \vspace{-0.5em}
        \( k \leftarrow k + 1 \)\;
    }
    \Return{\( \x^k \)\;}
\end{algorithm}

\paragraph{Finite termination property:}
For a convex quadratic function \( f: \R^d \to \R \), the conjugate gradient method terminates in at most \( d \) iterations.

\begin{theorem}{Conjugate gradient theorem:\\ Mutual \( \Q \)--conjugacy of directions in conjugate gradient method~\citep{Nocedal2006}}{}
    The directions \( \set{\u_k}_{k = 0}^{d - 1} \) generated by the conjugate gradient method are mutually \( \Q \)--conjugate.

    For all \( k \in \set{1, 2, \ldots, d - 1} \), we have
    \vspace{-0.5em}
    \begin{equation}\label{eq:conjugate-gradient-linear-spans}
        \spanset{\u_i}_{i = 0}^{k}
        =
        \spanset{\grad{f}{\x^i}}_{i = 0}^{k}
        =
        \spanset{\Q^i \, \grad{f}{\x^0}}_{i = 0}^{k}
    \end{equation}
    \begin{equation}\label{eq:conjugate-gradient-mutual-Q-conjugacy}
        \qf{\u_k}{\Q}[\u_i]
        = 0,
        \quad \forall i \in \set{0, 1, \ldots, k - 1}
    \end{equation}
\end{theorem}

\begin{proof}
    We prove this using mathematical induction.

    By the construction of \( \u_{k+1} \), we have that the successive directions \( \u_k, \u_{k+1} \) are \( \Q \)--conjugate, i.e.,
    \begin{equation}\label{eq:conjugate-gradient-successive-Q-conjugacy}
        \qf{\u_{k+1}}{\Q}[\u_k]
        = 0,
        \quad \forall k \in \set{0, 1, \ldots, d - 1}
    \end{equation}

    \begin{align*}
        \because
        \x^{k+1}
        & =
        \x^k + \alpha_k \u_k,
        \quad \forall k \in \set{0, 1, \ldots, d - 1}
        \\
        \implies
        \sum_{k = 0}^{d-1} \x^{k+1}
        & =
        \sum_{k = 0}^{d-1} \x^k
        + \sum_{k = 0}^{d-1} \alpha_k \u_k
        \implies
        \boxed{
            \x^d
            =
            \x^0
            + \sum_{k = 0}^{d-1} \alpha_k \u_k
        }
    \end{align*}

    \begin{align*}
        \because
        \grad{f}{\x}
        & =
        \Q \x + \h,
        \quad \forall \x \in \R^d
        \\
        \implies
        \grad{f}{\x^{k+1}}
        & =
        \Q \x^{k+1} + \h
        =
        \Q \big( \x^k + \alpha_k \u_k \big) + \h
        =
        \left( \Q \x^k + \h \right) + \alpha_k \Q \u_k
        \\ & =
        \grad{f}{\x^k} + \alpha_k \Q \u_k,
        \quad \forall k \in \set{0, 1, \ldots, d - 1}
        \\
        \implies
        \spanset{\Q \u_k, \grad{f}{\x^{k+1}}}
        & =
        \spanset{\Q \u_k, \grad{f}{\x^k} + \alpha_k \Q \u_k}
        =
        \spanset{\Q \u_k, \grad{f}{\x^k}}
    \end{align*}
    \begin{equation*}
        \therefore
        \boxed{
            \spanset{\Q \u_k, \grad{f}{\x^{k+1}}}
            =
            \spanset{\Q \u_k, \grad{f}{\x^k}},
            \quad \forall k \in \set{0, 1, \ldots, d - 1}
        }
    \end{equation*}

    \subsubsection*{Base case:}
    For \( k = 1 \), we have
    \begin{align*}
        \boxed{
            \spanset{\u_0, \u_1}
        }
        & =
        \spanset{\u_0, -\grad{f}{\x^1} + \beta_0 \u_0}
        =
        \spanset{\u_0, -\grad{f}{\x^1}}
        \\ & =
        \spanset{-\grad{f}{\x^0}, -\grad{f}{\x^1}}
        =
        \boxed{
            \spanset{\grad{f}{\x^0}, \grad{f}{\x^1}}
        }
        \\ & =
        \spanset{\grad{f}{\x^0}, \grad{f}{\x^0} + \alpha_0 \Q \u_0}
        =
        \spanset{\grad{f}{\x^0}, \Q \u_0}
        \\ & =
        \spanset{\grad{f}{\x^0}, -\Q \, \grad{f}{\x^0}}
        =
        \boxed{
            \spanset{\grad{f}{\x^0}, \Q \, \grad{f}{\x^0}}
        }
    \end{align*}
    and from~\peqref{eq:conjugate-gradient-successive-Q-conjugacy}, we have \( \qf{\u_1}{\Q}[\u_0] = 0 \), thereby proving the base case.

    Note that for \( k = 0 \), the first statement~\peqref{eq:conjugate-gradient-linear-spans} is trivial, since
    \begin{equation*}
        \spanset{\u_0}
        =
        \spanset{-\grad{f}{\x^0}}
        =
        \spanset{\grad{f}{\x^0}}
    \end{equation*}

    \subsubsection*{Inductive step:}
    Assume that the statements~\peqref{eq:conjugate-gradient-linear-spans} and~\peqref{eq:conjugate-gradient-mutual-Q-conjugacy} hold for some \( k = m, \; m < d - 1 \), i.e.,
    \begin{equation*}
        \spanset{\u_i}_{i = 0}^{m}
        =
        \spanset{\grad{f}{\x^i}}_{i = 0}^{m}
        =
        \spanset{\Q^i \, \grad{f}{\x^0}}_{i = 0}^{m}
    \end{equation*}
    \begin{equation*}
        \qf{\u_m}{\Q}[\u_i]
        = 0,
        \quad \forall i \in \set{0, 1, \ldots, m - 1}
    \end{equation*}

    We want to show that the statements hold for \( k = m + 1 \), i.e.,
    \begin{equation*}
        \spanset{\u_i}_{i = 0}^{m + 1}
        =
        \spanset{\grad{f}{\x^i}}_{i = 0}^{m + 1}
        =
        \spanset{\Q^i \, \grad{f}{\x^0}}_{i = 0}^{m + 1}
    \end{equation*}
    \begin{equation*}
        \qf{\u_{m + 1}}{\Q}[\u_i]
        = 0,
        \quad \forall i \in \set{0, 1, \ldots, m}
    \end{equation*}

    Now,
    \begin{align*}
        \because
        \grad{f}{\x^{m+1}}
        & =
        \grad{f}{\x^m} + \alpha_m \Q \u_m
        \implies
        \grad{f}{\x^{m+1}}
        \in
        \spanset{\grad{f}{\x^m}, \Q \u_m}
        \\
        \Q \u_{m+1}
        & \in
        \spanset{\Q^i \, \grad{f}{\x^0}}_{i = 0}^{m + 1}
        \\
        \grad{f}{\x^{m+1}}
        & \in
        \spanset{\Q^i \, \grad{f}{\x^0}}_{i = 0}^{m + 1},
        \quad \text{ from the inductive hypothesis}
        \\
        \implies
        \spanset{\grad{f}{\x^i}}_{i = 0}^{m + 1}
        & \subseteq
        \spanset{\Q^i \, \grad{f}{\x^0}}_{i = 0}^{m + 1}
        \\
        \because
        \Q^{m+1} \, \grad{f}{\x^0}
        & =
        \Q \pbrac{\Q^m \, \grad{f}{\x^0}}
        \in
        \spanset{\Q \u_i}_{i = 0}^{m}
        =
        \spanset{\grad{f}{\x^i}}_{i = 0}^{m + 1}
    \end{align*}
    \begin{equation*}
        \therefore
        \boxed{
            \spanset{\grad{f}{\x^i}}_{i = 0}^{m + 1}
            =
            \spanset{\Q^i \, \grad{f}{\x^0}}_{i = 0}^{m + 1}
        }
    \end{equation*}
    \begin{align*}
        \because
        \u_{m+1}
        & \in
        \spanset{\grad{f}{\x^{m+1}}, \u_m}
        \\
        \implies
        \spanset{\u_i}_{i = 0}^{m + 1}
        & =
        \spanset{\u_i}_{i = 0}^{m} + \spanset{\grad{f}{\x^{m+1}}}
    \end{align*}
    \begin{equation*}
        \therefore
        \boxed{
            \spanset{\u_i}_{i = 0}^{m + 1}
            =
            \spanset{\grad{f}{\x^i}}_{i = 0}^{m + 1}
        }
    \end{equation*}

    Now,
    \begin{align*}
        \qf{\u_{m+1}}{\Q}[\u_i]
        & =
        \qf{-\grad{f}{\x^{m+1}}}{\Q}[\u_i]
        + \beta_m \qf{\u_m}{\Q}[\u_i],
        \quad \forall i \in \set{0, 1, \ldots, m - 1}
        \\
        \because
        \qf{\u_m}{\Q}[\u_i]
        & = 0,
        \quad \text{ from the inductive hypothesis}
        \\
        \implies
        \qf{\u_{m+1}}{\Q}[\u_i]
        & =
        \qf{-\grad{f}{\x^{m+1}}}{\Q}[\u_i],
        \quad \forall i \in \set{0, 1, \ldots, m - 1}
        \\
        \because
        \Q \u_i
        & \in
        \spanset{\Q^j \, \grad{f}{\x^0}}_{j = 0}^{m}
        =
        \spanset{\u_j}_{j = 0}^{m},
        \quad \forall i \in \set{0, 1, \ldots, m - 1}
        \\
        \implies
        \Q \u_i
        & =
        \sum_{j = 0}^{m} \delta_j \, \u_j,
        \quad \delta_j \in \R,
        \; \forall i \in \set{0, 1, \ldots, m - 1}
        \\
        \implies
        \qf{\u_{m+1}}{\Q}[\u_i]
        & =
        \sum_{j = 0}^{m} \delta_j \, \dotp{\grad{f}{\x^{m+1}}}{\u_j},
        \quad \forall i \in \set{0, 1, \ldots, m - 1}
    \end{align*}

    From the expanding subspace theorem~\pthmref{thm:expanding-subspace-theorem}~\peqref{eq:conjugate-direction-orthogonality}, we have
    \begin{equation*}
        \dotp{\grad{f}{\x^{m+1}}}{\u_i}
        = \zero,
        \quad \forall i \in \set{0, 1, \ldots, m}
    \end{equation*}
    \begin{equation*}
        \implies
        \qf{\u_{m+1}}{\Q}[\u_i]
        = 0,
        \quad \forall i \in \set{0, 1, \ldots, m - 1}
    \end{equation*}

    From~\peqref{eq:conjugate-gradient-successive-Q-conjugacy}, we also have that \( \qf{\u_{m+1}}{\Q}[\u_m] = 0 \).
    Thereby, we finally have
    \begin{equation*}
        \therefore
        \boxed{
            \qf{\u_{m+1}}{\Q}[\u_i]
            = 0,
            \quad \forall i \in \set{0, 1, \ldots, m}
        }
    \end{equation*}

    Hence, we have shown that the statements hold for \( k = m + 1 \), thereby completing the inductive step.
\end{proof}

\begin{corollary}{{}~\citep{Nocedal2006}}{}
    The \( k \)-th direction \( \u_k \) generated by the conjugate gradient method at iteration \( k \) lies in the Krylov subspace of order \( k \) generated by \( \Q \) and the initial gradient \( \grad{f}{\x^0} \), i.e.,
    \vspace{-0.5em}
    \begin{equation*}
        \u_k
        \in
        \mathcal{K}_k(\Q, \nabla f(\x^0))
        =
        \spanset{\Q^i \, \grad{f}{\x^0}}_{i = 0}^{k - 1}
    \end{equation*}
\end{corollary}

\begin{algorithm}[H]\label{alg:practical-conjugate-gradient-method}
    \caption{
        {}~\citep{Nocedal2006}
        (Practical) Conjugate gradient algorithm for unconstrained minimisation of a convex quadratic function \( f: \R^d \to \R \).
        \vspace{-1em}
        \begin{equation*}
            \func{f}{\x}
            =
            \half \qf{\x}{\Q} + \dotp{\h}{\x},
            \quad \Q \in \PD, \; \h \in \R^d
        \end{equation*}
    }
    \SetAlgoLined{}
    \KwIn{
        First-order oracle for the objective function \( \func{f}{\x} \);
        Initial point \( \x^0 \in \R^d \);
    }
    \KwOut{
        Exact solution to the unconstrained minimisation problem \( \displaystyle \argmin_{\x \in \R^d} \func{f}{\x} \)\;
    }

    \( k \leftarrow 0 \)\;

    \( \u_0 \leftarrow -\grad{f}{\x^0} \)\;

    \While{\( k < d \) \emph{ AND } \( \x^k \) is not optimal}{
        Choose step length \( \alpha_k \) as
        \vspace{-0.5em}
        \begin{equation*}
            \alpha_k
            =
            \frac{\norm{\grad{f}{\x^k}}_2^2}{\qf{\u_k}{\Q}}
        \end{equation*}

        \vspace{-0.5em}
        Update the current point: \( \x^{k+1} = \x^k + \alpha_k \u_k \)\;

        Find the new gradient by using either the first-order oracle or the recurrence relation
        \vspace{-0.5em}
        \begin{equation*}
            \grad{f}{\x^{k+1}}
            =
            \grad{f}{\x^k}
            + \alpha_k \Q \u_k
        \end{equation*}
        or the quadratic structure of \( f \)
        \vspace{-0.5em}
        \begin{equation*}
            \grad{f}{\x^{k+1}}
            =
            \Q \x^{k+1} + \h
        \end{equation*}

        Find the next conjugate direction as
        \vspace{-0.5em}
        \begin{align*}
            \u_{k+1}
            & =
            -\grad{f}{\x^{k+1}}
            + \beta_k \u_k,
            \\
            \text{where }
            \beta_k
            & =
            \frac{\norm{\grad{f}{\x^{k+1}}}_2^2}{\norm{\grad{f}{\x^k}}_2^2}
        \end{align*}

        \vspace{-0.5em}
        \( k \leftarrow k + 1 \)\;
    }
    \Return{\( \x^k \)\;}
\end{algorithm}

\begin{theorem}{{}~\citep{Nocedal2006}}{}
    The algorithms given in~\palgref{alg:preliminary-conjugate-gradient-method} and~\palgref{alg:practical-conjugate-gradient-method} are mathematically equivalent, under exact arithmetic.
\end{theorem}

\begin{proof}
    The only differences between the two algorithms are in the equations used to compute \( \alpha_k \) and \( \beta_k \).

    The equivalence of \( \alpha_k \)'s follows from the expanding subspace theorem~\pthmref{thm:expanding-subspace-theorem}, as
    \begin{align*}
        \because
        \u_k
        & =
        -\grad{f}{\x^k}
        + \beta_{k - 1} \u_{k - 1},
        \quad \forall k \in \set{1, 2, \ldots, d - 1}
        \\
        \implies
        - \dotp{\grad{f}{\x^k}}{\u_k}
        & =
        \norm{\grad{f}{\x^k}}_2^2
        - \beta_{k - 1} \dotp{\grad{f}{\x^k}}{\u_{k-1}},
        \quad \forall k \in \set{1, 2, \ldots, d - 1}
        \\
        \because
        \dotp{\grad{f}{\x^k}}{\u_{k-1}}
        & = 0,
        \quad \forall k \in \set{1, 2, \ldots, d},
        \qquad \text{ from~\peqref{eq:conjugate-direction-orthogonality}}
        \\
        \implies
        - \dotp{\grad{f}{\x^k}}{\u_k}
        & =
        \norm{\grad{f}{\x^k}}_2^2,
        \quad \forall k \in \set{1, 2, \ldots, d - 1}
        \\
        \text{Also, }
        \because
        \u_0
        & =
        -\grad{f}{\x^0}
        \implies
        - \dotp{\grad{f}{\x^0}}{\u_0}
        =
        \norm{\grad{f}{\x^0}}_2^2
        \\
        \implies
        - \dotp{\grad{f}{\x^k}}{\u_k}
        & =
        \norm{\grad{f}{\x^k}}_2^2,
        \quad \forall k \in \set{0, 1, \ldots, d - 1}
        \\
        \because
        \qf{\u_k}{\Q}
        & > 0,
        \quad \forall k \in \set{0, 1, \ldots, d - 1},
        \qquad \because \Q \succ \zero
    \end{align*}
    \begin{equation}\label{eq:conjugate-gradient-step-length-equivalence}
        \therefore
        \boxed{
            \alpha_k
            =
            - \frac{\dotp{\grad{f}{\x^k}}{\u_k}}{\qf{\u_k}{\Q}}
            =
            \frac{\norm{\grad{f}{\x^k}}_2^2}{\qf{\u_k}{\Q}},
            \quad \forall k \in \set{0, 1, \ldots, d - 1}
        }
    \end{equation}

    \begin{align*}
        \because
        \grad{f}{\x^{k+1}}
        & =
        \grad{f}{\x^k}
        + \alpha_k \Q \u_k,
        \quad \forall k \in \set{0, 1, \ldots, d - 1}
        \\
        \implies
        \norm{\grad{f}{\x^{k+1}}}_2^2
        & =
        \norm{\grad{f}{\x^k}}_2^2
        + 2 \alpha_k \dotp{\grad{f}{\x^k}}{\Q \u_k}
        + \alpha_k^2 \norm{\Q \u_k}_2^2,
        \quad \forall k \in \set{0, 1, \ldots, d - 1}
        \\
        \text{Now, }
        \because
        \grad{f}{\x^k}
        & =
        -\u_k
        + \beta_{k - 1} \u_{k - 1},
        \quad \forall k \in \set{1, 2, \ldots, d - 1}
        \\
        \implies
        \qf{\grad{f}{\x^k}}{\Q}[\u_k]
        & =
        -\qf{\u_k}{\Q}
        + \beta_{k - 1} \cancel{ \qf{\u_{k-1}}{\Q}[\u_k] },
        \quad \forall k \in \set{1, 2, \ldots, d - 1}
        \\
        & =
        -\qf{\u_k}{\Q},
        \qquad \text{ from~\peqref{eq:conjugate-gradient-successive-Q-conjugacy}}
        \\
        \text{Also, }
        \because
        \u_0
        & =
        -\grad{f}{\x^0}
        \implies
        \qf{\grad{f}{\x^0}}{\Q}[\u_0]
        =
        -\qf{\u_0}{\Q}
        \\
        \implies
        \qf{\grad{f}{\x^k}}{\Q}[\u_k]
        & =
        -\qf{\u_k}{\Q},
        \quad \forall k \in \set{0, 1, \ldots, d - 1}
        \\
        \implies
        \norm{\grad{f}{\x^{k+1}}}_2^2
        & =
        \norm{\grad{f}{\x^k}}_2^2
        - 2 \alpha_k \qf{\u_k}{\Q}
        + \alpha_k^2 \norm{\Q \u_k}_2^2,
        \quad \forall k \in \set{0, 1, \ldots, d - 1}
        \\
        \because
        \alpha_k \qf{\u_k}{\Q}
        & =
        \norm{\grad{f}{\x^k}}_2^2,
        \quad \forall k \in \set{0, 1, \ldots, d - 1},\qquad \text{ from~\peqref{eq:conjugate-gradient-step-length-equivalence}}
        \\
        \implies
        \norm{\grad{f}{\x^{k+1}}}_2^2
        & =
        \cancel{ \norm{\grad{f}{\x^k}}_2^2 }
        - \cancel{2} \norm{\grad{f}{\x^k}}_2^2
        + \alpha_k^2 \norm{\Q \u_k}_2^2,
        \quad \forall k \in \set{0, 1, \ldots, d - 1}
        \\ & =
        - \norm{\grad{f}{\x^k}}_2^2
        + \alpha_k^2 \norm{\Q \u_k}_2^2,
        \quad \forall k \in \set{0, 1, \ldots, d - 1}
        \\
        \text{Now, }
        \qf{\grad{f}{\x^{k+1}}}{\Q}[\u_k]
        & =
        \qf{\grad{f}{\x^k}}{\Q}[\u_k]
        + \alpha_k \norm{\Q \u_k}_2^2,
        \quad \forall k \in \set{0, 1, \ldots, d - 1}
        \\ & =
        -\qf{\u_k}{\Q}
        + \alpha_k \norm{\Q \u_k}_2^2,
        \quad \forall k \in \set{0, 1, \ldots, d - 1}
        \\
        \implies
        \alpha_k \norm{\Q \u_k}_2^2
        & =
        \qf{\grad{f}{\x^{k+1}}}{\Q}[\u_k]
        + \qf{\u_k}{\Q},
        \quad \forall k \in \set{0, 1, \ldots, d - 1}
        \\
        \implies
        \norm{\grad{f}{\x^{k+1}}}_2^2
        & =
        - \norm{\grad{f}{\x^k}}_2^2
        + \alpha_k \qf{\grad{f}{\x^{k+1}}}{\Q}[\u_k]
        + \alpha_k \qf{\u_k}{\Q},
        \quad \forall k \in \set{1, 2, \ldots, d - 1}
        \\ & =
        - \cancel{ \norm{\grad{f}{\x^k}}_2^2 }
        + \alpha_k \qf{\grad{f}{\x^{k+1}}}{\Q}[\u_k]
        + \cancel{ \norm{\grad{f}{\x^k}}_2^2 },
        \quad \forall k \in \set{1, 2, \ldots, d - 1}
        \\
        \implies
        &
        \boxed{
            \norm{\grad{f}{\x^{k+1}}}_2^2
            =
            \alpha_k \qf{\grad{f}{\x^{k+1}}}{\Q}[\u_k],
            \quad \forall k \in \set{0, 1, \ldots, d - 1}
        }
        \\
        \implies
        \norm{\grad{f}{\x^{k+1}}}_2^2
        & =
        \frac{\norm{\grad{f}{\x^k}}_2^2}{\qf{\u_k}{\Q}} \qf{\grad{f}{\x^{k+1}}}{\Q}[\u_k],
        \quad \forall k \in \set{0, 1, \ldots, d - 1}
    \end{align*}
    \begin{equation*}
        \therefore
        \boxed{
            \beta_k
            =
            \frac{\norm{\grad{f}{\x^{k+1}}}_2^2}{\norm{\grad{f}{\x^k}}_2^2}
            =
            \frac{\qf{\grad{f}{\x^{k+1}}}{\Q}[\u_k]}{\qf{\u_k}{\Q}},
            \quad \forall k \in \set{0, 1, \ldots, d - 1}
        }
    \end{equation*}
\end{proof}

\section{Polynomial optimisation perspective}

Define the polynomial \( P_k^\ast: \bbrac{\;\cdot\;} \to \R \) of degree at most \( k \) as
\begin{equation*}
    P_k^\ast(x)
    \triangleq
    \sum_{i = 0}^{k} \gamma_i \, x^i,
    \quad \gamma_i \in \R,
    \; \forall i \in \set{0, 1, \ldots, k},
    \; \forall k \in \set{0, 1, \ldots, d - 1}
\end{equation*}

\begin{align*}
    \implies
    \x^{k+1}
    & =
    \x^0 + \sum_{i = 0}^{k} \alpha_i \, \u_i
    =
    \x^0 + \sum_{i = 0}^{k} \gamma_i \, \Q^i \, \grad{f}{\x^0}
    \\
    \implies
    &
    \boxed{
        \x^{k+1}
        =
        \x^0 + P_k^\ast(\Q) \, \grad{f}{\x^0},
        \quad \forall k \in \set{0, 1, \ldots, d - 1}
    }
\end{align*}

\begin{align*}
    \implies
    f(\x) - f(\xstar)
    & =
    \pbrac{\half \qf{\x}{\Q} + \dotp{\h}{\x}} - \pbrac{\half \qf{\xstar}{\Q} + \dotp{\h}{\xstar}}
    \\ & =
    \half \qf{\x}{\Q} - \qf{\x}{\Q}[\xstar] + \half \qf{\xstar}{\Q}
    =
    \half \qf{(\x - \xstar)}{\Q}
\end{align*}
\begin{equation*}
    \implies
    \boxed{
        \half \norm{\x - \xstar}^2_{\Q}
        =
        f(\x) - f(\xstar),
        \quad \forall \x \in \R^d
    }
\end{equation*}

\begin{align*}
    \implies
    \x^{k+1} - \xstar
    & =
    \x^0 + P_k^\ast(\Q) \, \grad{f}{\x^0} - \xstar
    \\ & =
    (\x^0 - \xstar) + P_k^\ast(\Q) \, \Q \, (\x^0 - \xstar)
    =
    \bbrac{\I + P_k^\ast(\Q) \, \Q} \, (\x^0 - \xstar)
\end{align*}

\begin{equation*}
    \implies
    P_k^\ast
    =
    \argmin_{P_k}
    \norm{\x^0 + P_k(\Q) \, \grad{f}{\x^0} - \xstar}_{\Q}
\end{equation*}

From spectral decomposition theorem~\pthmref{thm:spectral-decomposition-theorem}, we have
\(
    \Q
    =
    \sum_{i = 1}^{d} \lambda_i \, \outp{\v_i}{\v_i}
\).

Since \( \set{\v_i}_{i = 1}^{d} \) forms an orthonormal basis for \( \R^d \), we have
\begin{align*}
    \implies
    \x^0 - \xstar
    & =
    \sum_{i = 1}^{d} \mu_i \, \v_i,
    \quad \mu_i \in \R,
    \; \forall i \in \set{1, 2, \ldots, d}
    \implies
    \norm{\x^0 - \xstar}_{\Q}^2
    =
    \sum_{i = 1}^{d} \lambda_i \, \mu_i^2
    \\
    P_k(\Q) \, \v_i
    & =
    P_k(\lambda_i) \, \v_i,
    \quad \forall i \in \set{1, 2, \ldots, d}
    \\
    \implies
    \x^{k+1} - \xstar
    & =
    \bbrac{\I + P_k^\ast(\Q) \, \Q} \, (\x^0 - \xstar)
    =
    \sum_{i = 1}^{d} \mu_i \, \bbrac{1 + P_k^\ast(\lambda_i) \, \lambda_i} \, \v_i
    \\
    \implies
    \norm{\x^{k+1} - \xstar}_{\Q}^2
    & =
    \sum_{i = 1}^{d} \lambda_i \, \mu_i^2 \, \bbrac{1 + P_k^\ast(\lambda_i) \, \lambda_i}^2
    =
    \min_{P_k}
    \sum_{i = 1}^{d} \lambda_i \, \mu_i^2 \, \bbrac{1 + P_k(\lambda_i) \, \lambda_i}^2
    \\
    \implies
    \norm{\x^{k+1} - \xstar}_{\Q}^2
    & \leq
    \min_{P_k}
    \max_{1 \leq i \leq d}
    \bbrac{1 + P_k(\lambda_i) \, \lambda_i}^2
    \sum_{i = 1}^{d} \lambda_i \, \mu_i^2
\end{align*}
\begin{equation*}
    \therefore
    \boxed{
        \norm{\x^{k+1} - \xstar}_{\Q}^2
        \leq
        \min_{P_k}
        \max_{1 \leq i \leq d}
        \bbrac{1 + P_k(\lambda_i) \, \lambda_i}^2
        \norm{\x^0 - \xstar}_{\Q}^2
    }
\end{equation*}
