\chapter{Introduction}

\section{Historical problems}

\subsection{Heron's problem}

\begin{figure}[h]
    \centering
    \begin{minipage}{0.6\textwidth}
        \centering
        \includegraphics[width=\textwidth]{figures/heron/_}
    \end{minipage}%
    \hfill
    \begin{minipage}{0.4\textwidth}
        \paragraph{Given:}
        Two points \( A \) and \( B \) on the same side of a line \( L \).

        \paragraph{To find:}
        A point \( P \) on line \( L \) such that the total distance \( AP + PB \) is minimized.

        \paragraph{Solution:}
        Reflect one of the points, say \( A \), across line \( L \) to get point \( A' \).
        The optimal point \( P^* \) is the intersection of line \( L \) and the line segment \( A'B \).
        \begin{equation*}
            \min_{P \in L} \; AP + PB = \min_{P \in L} \; A'P + PB = A'B
        \end{equation*}
    \end{minipage}
\end{figure}

\subsection{Stigler's diet problem}

\begin{minipage}{0.4\textwidth}
    \begin{align*}
        \min_{x} \quad & \sum_{i} c_i x_i \\
        \text{subject to} \quad
        & \sum_{i} x_i a_{ij} \geq r_j, \quad \forall j \\
        & x_i \geq 0, \quad \forall i
    \end{align*}
\end{minipage}
\hfill
\begin{minipage}{0.55\textwidth}
    where
    \begin{description}[align=right,labelwidth=1.2em,labelsep=0.8em]
        \item[\( i \)] Index for food items.

        \item[\( j \)] Index for nutrients.

        \item[\( x_i \)] Quantity of food item \( i \) to include in the diet.

        \item[\( c_i \)] Cost per unit of food item \( i \).

        \item[\( a_{ij} \)] Amount of nutrient \( j \) in one unit of food item \( i \).

        \item[\( r_j \)] Minimum required amount of nutrient \( j \) per day.
    \end{description}
\end{minipage}

\section{Optimisation problem}

An \textbf{unconstrained optimisation problem} is the problem of finding a point \( \mathbf{x}^* \in \mathbb{R}^d \) that minimises a function \( f: \mathbb{R}^d \to \mathbb{R} \), i.e.,
\begin{equation*}
    f(\mathbf{x}^*) = \min_{\mathbf{x} \in \mathbb{R}^d} f(\mathbf{x})
    \quad \iff \quad
    \mathbf{x}^* = \argmin_{\mathbf{x} \in \mathbb{R}^d} f(\mathbf{x})
    \quad \iff \quad
    f(\mathbf{x}^*) \leq f(\mathbf{x}), \quad \forall \mathbf{x} \in \mathbb{R}^d
\end{equation*}

The function \( f \) is called the \textbf{objective function}, the point \( \mathbf{x}^* \) is called the \textbf{minimum point}, and the value \( f(\mathbf{x}^*) \) is called the \textbf{minimum value}.

\newpage
\subsection{Examples}

% chktex-file 44
\begin{table}[h!]
    \centering
    \renewcommand{\arraystretch}{1.5}
    \begin{tabular}{|l|l|l|}
        \hline
        \textbf{Objective function \( f(x) \)}
        & \textbf{Minimum point \( x^* \)}
        & \textbf{Minimum value \( f(x^*) \)} \\
        \hline

        \( f(x) = \frac{1}{2}x^2, \quad x \in \mathbb{R} \)
        & \( x^* = 0 \)
        & \( f(x^*) = 0 \) \\
        \hline

        \( f(x) = \frac{1}{2}{(x - x_0)}^2, \quad x, x_0 \in \mathbb{R} \)
        & \( x^* = x_0 \)
        & \( f(x^*) = 0 \) \\
        \hline

        \( f(x) = \frac{1}{2}{(x - x_0)}^2 + c, \quad x, x_0, c \in \mathbb{R} \)
        & \( x^* = x_0 \)
        & \( f(x^*) = c \) \\
        \hline

        \hypertarget{1D-quadratic}{
            \( f(x) = ax^2 + bx + c, \quad x, a, b, c \in \mathbb{R}, \; a > 0 \)
        }
        & \( x^* = -\cfrac{b}{2a} \)
        & \( f(x^*) = c - \cfrac{b^2}{4a} \)
        \rule[-13pt]{0pt}{36pt} \\
        \hline

        \( f(t) = \cfrac{1}{2} {\big\Vert \mathbf{u} - t \mathbf{v} \big\Vert}^2, \quad t \in \mathbb{R}, \; \mathbf{u}, \mathbf{v} \in \mathbb{R}^d, \; \mathbf{v} \neq \mathbf{0} \)
        & \( t^* = \cfrac{\mathbf{u}^\top \mathbf{v}}{\mathbf{v}^\top \mathbf{v}} \)
        & \( f(t^*) = \cfrac{1}{2} \left( {\Vert \mathbf{u} \Vert}^2 - \cfrac{{(\mathbf{u}^\top \mathbf{v})}^2}{\mathbf{v}^\top \mathbf{v}} \right) \)
        \rule[-20pt]{0pt}{48pt} \\
        \hline
    \end{tabular}
\end{table}

Note that \( \because f(t) \geq 0, \; \forall t \in \mathbb{R} \implies f(t^*) \geq 0 \), which gives the Cauchy-Schwarz inequality~\psecref{sec:cauchy-schwarz-inequality}.
